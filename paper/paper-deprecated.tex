%-----------------------------------------------------------------------------
%               Template for OOPSLA
%               based on:
%               Template for sigplanconf LaTeX Class
%
% Name:         sigplanconf-template.tex
%
% Purpose:      A template for sigplanconf.cls, which is a LaTeX 2e class
%               file for SIGPLAN conference proceedings.
%
% Guide:        Refer to "Author's Guide to the ACM SIGPLAN Class,"
%               sigplanconf-guide.pdf
%
% Author:       Paul C. Anagnostopoulos
%               Windfall Software
%               978 371-2316
%               paul@windfall.com
%
% Created:      15 February 2005
%
%-----------------------------------------------------------------------------
\documentclass[runningheads,a4paper]{llncs}
%\documentclass[sigconf,authordraft]{acmart}
%\acmConference[ESEC/FSE 2017]{11th Joint Meeting of the European Software Engineering Conference and the ACM SIGSOFT Symposium on the Foundations of Software Engineering}{4--8 September, 2017}{Paderborn, Germany}
%\documentclass[10pt,numbers]{sigplanconf}
%\usepackage[1stsubmission]{oopsla2016}
%\usepackage[2ndsubmission]{oopsla2016}

%\usepackage[scaled]{helvet} % see www.ctan.org/get/macros/latex/required/psnfss/psnfss2e.pdf

\usepackage{microtype}
\usepackage{url}                  % format URLs
\usepackage{listings}          % format code
\lstset{
  mathescape, 
  language={Java},
  basicstyle=\footnotesize
}
\usepackage{enumitem}      % adjust spacing in enums
%\usepackage[colorlinks=true,allcolors=blue,breaklinks,draft=false]{hyperref}   % hyperlinks, including DOIs and URLs in bibliography
% known bug: http://tex.stackexchange.com/questions/1522/pdfendlink-ended-up-in-different-nesting-level-than-pdfstartlink

\usepackage{graphicx}
\usepackage{float,subfig}
\usepackage{xspace,framed}
\usepackage{colortbl}
\usepackage{calc}
 
\usepackage{algorithm}
\usepackage{amsfonts}
\usepackage{amsmath}
\usepackage{multicol}
\usepackage{multirow}
\usepackage{tikz}
\usepackage[justification=centering]{caption}
\usepackage{stmaryrd}
\usetikzlibrary{positioning, arrows, automata, shapes}
\usepackage{hhline}
\usepackage{pifont}
\usepackage{cite}
\usepackage{pdflscape} % Experiments table is landscape
\usepackage{longtable}
\usepackage{afterpage}
\usepackage{wasysym}

\usepackage[justification=centering]{caption}

\newcommand{\doi}[1]{}{}%{doi:~\href{http://dx.doi.org/#1}{\Hurl{#1}}}   % print a hyperlinked DOI

\newcommand{\cbmc}{CBMC}
\newcommand{\sat}{SAT}

% Theorems and such

%\newtheorem{definition}{Definition}
%\newtheorem{lemma}{Lemma}
%\newtheorem{example}{Example}
%\newtheorem{theorem}{Theorem}

\newcommand{\defeq}{\ensuremath{\stackrel{\mathrm{def}}{=}}}

% \algrenewcommand\algorithmicrequire{\textbf{Precondition:}}
% \algrenewcommand\algorithmicensure{\textbf{Postcondition:}}


\newcommand{\univ}{\ensuremath{\mathcal{P}}\xspace}
\newcommand{\ex}{\ensuremath{\mathcal{E}}\xspace}
\newcommand{\hlen}{\ensuremath{\mathit{len}}\xspace}
\newcommand{\hskips}{\ensuremath{\mathit{skip}}\xspace}
\newcommand{\hlimit}{\ensuremath{\mathit{limit}}\xspace}
\newcommand{\hmax}{\ensuremath{\mathit{max}}\xspace}
\newcommand{\hmin}{\ensuremath{\mathit{min}}\xspace}
\newcommand{\hforeach}{\ensuremath{\mathit{foreach}}\xspace}
\newcommand{\hmap}{\ensuremath{\mathit{map}}\xspace}
\newcommand{\hsiz}{\ensuremath{\mathit{size}}\xspace}
\newcommand{\hval}{\ensuremath{\mathit{val}}\xspace}
\newcommand{\hnodes}{\ensuremath{\mathit{nodes}}\xspace}
\newcommand{\hedges}{\ensuremath{\mathit{edges}}\xspace}

% Logic operators

\newcommand{\halias}{\ensuremath{\mathit{alias}}\xspace}
\newcommand{\heapsize}{\ensuremath{\mathit{size}}\xspace}
\newcommand{\hispath}{\ensuremath{\mathit{isPath}}\xspace}
\newcommand{\hfilter}{\ensuremath{\mathit{filter}}\xspace}
\newcommand{\hnew}{\ensuremath{\mathit{new}}\xspace}
\newcommand{\hassign}{\ensuremath{\mathit{assign}}\xspace}
\newcommand{\halloc}{\ensuremath{\mathit{alloc}}\xspace}
\newcommand{\hset}{\ensuremath{\mathit{set}}\xspace}
\newcommand{\hget}{\ensuremath{\mathit{get}}\xspace}
\newcommand{\hremove}{\ensuremath{\mathit{remove}}\xspace}
\newcommand{\hadd}{\ensuremath{\mathit{add}}\xspace}

% Logic extension operators

\newcommand{\iterators}{\ensuremath{\mathit{I}}\xspace}
\newcommand{\lists}{\ensuremath{\mathit{L}}\xspace}

\newcommand{\id}{\ensuremath{\mathrm{id}}\xspace}
\newcommand{\hsubdivide}{\ensuremath{\mathit{subdivide}}\xspace}
\newcommand{\hsmoothen}{\ensuremath{\mathit{smoothen}}\xspace}
\newcommand{\hfresh}{\ensuremath{\mathit{fresh}()}\xspace}
\newcommand{\hexists}{\ensuremath{\mathit{exists}}\xspace}
\newcommand{\hforall}{\ensuremath{\mathit{forall}}\xspace}
\newcommand{\hsorted}{\ensuremath{\mathit{sorted}}\xspace}
\newcommand{\hls}[2]{#1 {\rightarrow^*} #2}
\newcommand{\hnull}{\ensuremath{\mathbf{null}}}
\newcommand{\sem}[1]{\ensuremath{\llbracket{#1}\rrbracket}}
% \newcommand{\xmark}{\ding{55}}

\newcommand*\Let[2]{\State #1 $\gets$ #2}

\newcommand{\bv}[2]{\mathcal{BV}(#1, #2)}

\newcommand{\todo}[1]{{\color{red} TODO: {#1}}} 

\iffalse
\lstset{emph={
  assume, assert}, emphstyle=\bfseries}
\fi


% Logic things

\newcommand{\true}{\textit{true}\xspace}
\newcommand{\false}{\textit{false}\xspace}
\newcommand{\unknown}{\ensuremath{\mathsf{?}}\xspace}

% Heap variables

\newcommand{\numNodes}{\ensuremath{N}\xspace}
\newcommand{\numPointers}{\ensuremath{|PV|}\xspace}
\newcommand{\numPositional}{\ensuremath{NPos}\xspace}
\newcommand{\numPredicates}{\ensuremath{\mathsf{NPred}}\xspace}
\newcommand{\allocatedNodes}{\ensuremath{\mathsf{NAlloc}}\xspace}

% Java collections section

\newcommand{\listToPtr}[1]{\ensuremath{#1^{\mathcal{P}}}\xspace}
\newcommand{\preHeap}{\ensuremath{h}\xspace}
\newcommand{\postHeap}{\ensuremath{h'}\xspace}
\newcommand{\freshP}{\ensuremath{p}\xspace}

\newcommand{\hsucc}{\ensuremath{succ}\xspace}
\newcommand{\atindex}{\ensuremath{@}\xspace}
\newcommand{\swapNext}{\ensuremath{swap}\xspace}
\newcommand{\updateI}{\ensuremath{upI}\xspace}
\newcommand{\updateL}{\ensuremath{upL}\xspace}

\newcommand{\hprev}{\ensuremath{\mathit{prev}}\xspace}

% Java operators

\newcommand{\java}[1]{\ensuremath{\mathtt{#1}}\xspace}
\newcommand{\jadd}{\ensuremath{\java{add}}\xspace}
\newcommand{\jlist}{\ensuremath{\java{ArrayList}}\xspace}
\newcommand{\jnew}{\ensuremath{\java{new}}\xspace}
\newcommand{\jset}{\ensuremath{\java{set}}\xspace}
\newcommand{\jcontains}{\ensuremath{\java{contains}}\xspace}
\newcommand{\jcontainsall}{\ensuremath{\java{containsAll}}\xspace}
\newcommand{\jequals}{\ensuremath{\java{equals}}\xspace}
\newcommand{\jindexof}{\ensuremath{\java{indexOf}}\xspace}
\newcommand{\jlastindexof}{\ensuremath{\java{lastIndexOf}}\xspace}
\newcommand{\jremove}{\ensuremath{\java{remove}}\xspace}
\newcommand{\jremoveall}{\ensuremath{\java{removeAll}}\xspace}
\newcommand{\jretainall}{\ensuremath{\java{retainAll}}\xspace}
\newcommand{\jaddall}{\ensuremath{\java{addAll}}\xspace}
\newcommand{\jtoarray}{\ensuremath{\java{toArray}}\xspace}
\newcommand{\jclear}{\ensuremath{\java{clear}}\xspace}
\newcommand{\jget}{\ensuremath{\java{get}}\xspace}
\newcommand{\jhashcode}{\ensuremath{\java{hashCode}}\xspace}
\newcommand{\jisempty}{\ensuremath{\java{isEmpty}}\xspace}
\newcommand{\jiterator}{\ensuremath{\java{iterator}}\xspace}
\newcommand{\jlistiterator}{\ensuremath{\java{listIterator}}\xspace}
\newcommand{\jsize}{\ensuremath{\java{size}}\xspace}
\newcommand{\jsublist}{\ensuremath{\java{subList}}\xspace}
\newcommand{\jhasnext}{\ensuremath{\java{hasNext}}\xspace}
\newcommand{\jnext}{\ensuremath{\java{next}}\xspace}


% Helper formulas



% implementation data-structure variables

\newcommand{\code}[1]{\lstinline|#1|} % TODO: This is barely visible!
\newcommand*{\codenospace}[1]{\lstinline|#1|\kern-1ex}%
\newcommand{\ptr}{\lstinline|ptr|}
\newcommand{\succs}{\lstinline|succ|}
\newcommand{\prev}{\lstinline|prev|}
\newcommand{\data}{\lstinline|data|}
\newcommand{\val}{\lstinline|val|}
\newcommand{\dist}{\lstinline|dist|}
\newcommand{\pred}{\lstinline|pred|}
\newcommand{\sorted}{\lstinline|sorted|}
\newcommand{\upperBound}{\lstinline|ub|}
\newcommand{\lowerBound}{\lstinline|lb|}

% Name of the logic

\newcommand{\logic}{JST\xspace}
\newcommand{\tool}{Kayak\xspace}
\newcommand{\badlogic}{JCA+sharing\xspace}

\newcommand{\equivState}{\ensuremath{\mathit{equivState}}\xspace}
\newcommand{\heapEquiv}{\ensuremath{\mathit{heapEquiv}}\xspace}

% Results

\newcommand{\cmark}{\ding{51}}%
\newcommand{\xmark}{\ding{55}}%
\newcommand{\bench}[1]{{\footnotesize\textsf{#1}}\xspace}

%%% Local Variables:
%%% mode: latex
%%% TeX-master: "paper"
%%% End:


\tikzset{
    %Define style for boxes
    box/.style={
           rectangle,
           rounded corners,
           draw=black, very thick,
           fill=yellow!20,
           text width=10em,
           minimum height=2em,
           text centered},
 % 
  futurebox/.style={
           rectangle,
           rounded corners,
           draw=black, very thick,
           fill=blue!10,
           text width=10em,
           minimum height=2em,
           text centered},
 % 
  longfuturebox/.style={
           rectangle,
           rounded corners,
           draw=black, very thick,
           fill=purple!20,
           text width=10em,
           minimum height=2em,
           text centered},
%
    % Define arrow style
    arrow/.style={
           ->,
           thick,
           shorten <=2pt,
           shorten >=2pt,}
}


%\setlength{\belowcaptionskip}{-10pt}
\setlength{\textfloatsep}{1em}
\setlength{\dbltextfloatsep}{1em}
\setlength{\abovecaptionskip}{0.1em}
% \floatsep: space left between floats (12.0pt plus 2.0pt minus 2.0pt).
% \textfloatsep: space between last top float or first bottom float and the text (20.0pt plus 2.0pt minus 4.0pt).
% \intextsep : space left on top and bottom of an in-text float (12.0pt plus 2.0pt minus 2.0pt).
% \dbltextfloatsep is \textfloatsep for 2 column output (20.0pt plus 2.0pt minus 4.0pt).
% \dblfloatsep is \floatsep for 2 column output (12.0pt plus 2.0pt minus 2.0pt).
% \abovecaptionskip: space above caption (10.0pt).
% \belowcaptionskip: space below caption (0.0pt).

\allowdisplaybreaks

%%%%%%%%%%%%%%%%%%%%%%%%%%%%%%%%%%%%%
%% uncomment for extended version 
\newcommand*{\extended}{}
%%%%%%%%%%%%%%%%%%%%%%%%%%%%%%%%%%%%%


% The following oopsla2016 options are available:
%
% 1stsubmission   For the initial submission
% 2ndsubmission   For the 2nd submission
% final           For camera-ready

\begin{document}

%\copyrightdata{978-1-nnnn-nnnn-n/yy/mm} 
%\doi{nnnnnnn.nnnnnnn}

% Uncomment one of the following two, if you are not going for the 
% traditional copyright transfer agreement.

%\exclusivelicense                % ACM gets exclusive license to publish, 
                                  % you retain copyright

%\permissiontopublish             % ACM gets nonexclusive license to publish
                                  % (paid open-access papers, 
                                  % short abstracts)

\title{Deprecated elimination with program synthesis}
%\subtitle{Subtitle Text, if any}
% double-blind submission
% single-blind only for Technical Research: http://icse2017.gatech.edu/technical-research-cfp
 % \authorinfo{Cristina David}
 %            {University of Oxford}
 %            {cristina.david@cs.ox.ac.uk}
 % \authorinfo{Pascal Kesseli}
 %            {University of Oxford}
 %            {pascal.kesseli@cs.ox.ac.uk}
 % \authorinfo{Daniel Kroening}
 %            {University of Oxford}
 %            {kroening@cs.ox.ac.uk}

\author{
\mbox{Cristina David}\inst{1} \and
Pascal Kesseli\inst{2} \and
Daniel Kroening\inst{2}}

\institute{University of Bristol, UK 
\and
  University of Oxford, UK}

%% \author{Cristina David}
%% \affiliation{University of Oxford}
%% \email{cristina.david@cs.ox.ac.uk}

%% \author{Pascal Kesseli}
%% \affiliation{University of Oxford}
%% \email{pascal.kesseli@cs.ox.ac.uk}

%% \author{Daniel Kroening}
%% \affiliation{University of Oxford}
%% \email{kroening@cs.ox.ac.uk}

\maketitle

\begin{abstract}
%
  Refactorings are structured changes to existing software that leave its
  externally observable behaviour unchanged.  Their intent is to improve
  readability, performance or other non-behavioural properties. 
  State-of-the-art automatic refactoring tools are {\em syntax}-driven and,
  therefore, overly conservative.  In this paper we explore {\em
  semantics}-driven refactoring, which enables much more sophisticated
  refactoring schemata.  As~an exemplar of this broader idea, we present
  an automatic refactoring tool that replaces uses of deprecated legacy APIs/libraries
  with the latest ones.
  Our refactoring procedure performs
  semantic reasoning and search in the space of possible refactorings using
  automated program synthesis.  Our experimental results support the
  conjecture that semantics-driven refactorings are more precise and are
  able to rewrite more complex code scenarios when compared to syntax-driven
  refactorings.
%
\end{abstract}

%\keywords{program refactoring, program synthesis, program verification}

%\category{CR-number}{subcategory}{third-level}

% general terms are not compulsory anymore, 
% you may leave them out
%\terms
%term1, term2

%% \keywords
%% Maintenance and reuse



\section{Introduction}

As a project evolves, there are certain fields, methods or classes
that the developers are discouraged from using in the future as
they've been superseded and may cease to exist in the future.
However, removing them directly would break the backward compatibility
of the project's API.  Instead, such elements can be tagged with the
\code{@Deprecated} annotation.  In general, when deprecating a
field/method/class, the \code{@deprecated} Javadoc tag is used in the
comment section to inform the developer the reason of deprecation and
what can be used in place.

%% \subsection{Deprecated example}

%% The {\em Deprecated} annotation in Java is used to mark program elements which
%% programmers are discouraged from using for various reasons such as future API
%% changes or security problems in the method. Usually API designers provide
%% programmers with recommendations about what they should use instead.

Even in the presence of such recommendations, the transformation of
existing code to the recommended format is not always straightforward,
as illustrated in Fig.~\ref{ex:deprecated-method}. In the example, we
make use of the \code{Date} class, which is deprecated with the
recommendation to use \code{Calendar} instead.  However, the
\code{Date} constructor expects its year parameter to be an offset
from 1900, i.e. the year 2020 is expressed as 120.  This needs to be
transformed to 2020 when using \code{Calendar} API, as shown in the
example. Moreover, the \code{Calendar} constructor is protected,
meaning that instead we must use \code{getInstance} to obtain a
\code{Calendar} object, followed by setting the date using \code{set}
and returning a Date object by invoking \code{getTime}.
Alternatively, the \code{GregorianCalendar} class may be used with
similar difficulties.

\begin{figure}
\begin{lstlisting}[mathescape=true,showstringspaces=false]
void main(String[] args) {
  ByteArrayOutputStream bos = ...;
  // Deprecated:
  Date date = new Date(120, 12, 30);
 
  // Should have been:
  final Calendar calendar = Calendar.getInstance();
  calendar.set(2020, 12, 30);
  final Date date = calendar.getTime();

  // Or:
  final Date date = new GregorianCalendar(2020, 12, 30)
                      .getTime();
}
\end{lstlisting}
\caption{Deprecated method example.}
\label{ex:deprecated-method}
\end{figure}


While it is trivial for IDEs such as Eclipse or IntelliJ IDEA to
recognise the use of deprecated APIs and warn the programmer about it,
refactoring the code to a preferred API is significantly more
complex. Indeed neither Eclipse 4.15.0 nor IntelliJ IDEA 2020.2.2 make
any attempt to suggest a better alternative to the user. There is also
very limited support offered by the research community, mainly
focusing on replacing calls to deprecated methods by their
bodies~\cite{DBLP:conf/paste/Perkins05}. Consequently, the majority of
such refactorings are done manually, which is a costly,
time-intensive, and not least error-prone process.  This situation is
also a real issue for legacy code that makes use of deprecated APIs
that need to be replaced with latest APIs/libraries.

The main cause of difficulty when replacing deprecated instances lies
with the fact that such refactorings are based on semantics rather
than syntax. For illustration, let's refer back to the example in
Figure~\ref{ex:deprecated-method}.  A {\em syntax-driven refactoring}
addresses structural changes to the program requiring only limited
information about a program's semantics, whereas a {\em
  semantics-driven refactoring} requires detailed understanding of the
program semantics in order to be applied soundly.  Clearly, the
refactoring of the call to the deprecated \code{Date} constructor
falls into the latter category. %%, as we need to understand exactly

%\subsection{From the previous introduction:}
%% Refactorings are structured changes to existing software which leave its
%% externally observable behaviour unchanged.  They improve non-functional
%% properties of the program code, such as testability, maintainability and
%% extensibility while retaining the semantics of the program.  Ultimately,
%% refactorings can improve the design of code, help finding bugs as well
%% as increase development speed and are therefore seen as an integral part
%% of agile software engineering processes~\cite{DBLP:conf/xpu/Kerievsky04b,
%% Fowler1999}.

%% However, manual refactorings are a costly, time-intensive, and not least
%% error-prone process.  This has motivated work on automating specific
%% refactorings, which promises safe application to large code bases at low
%% cost.  We differentiate in this context between {\em syntax-driven} and {\em
%% semantics-driven} refactorings.  While the former address structural changes
%% to the program requiring only limited information about a program's
%% semantics, the latter require detailed understanding of the program
%% semantics in order to be applied soundly.  An example of a refactoring that
%% requires a semantics-driven approach is {\em Substitute Algorithm}, where an
%% algorithm is replaced by a clearer, but equivalent
%% version~\cite{Fowler1999}.
%
%AST-based constraints 
%
%% A syntax-driven approach is insufficient to perform such substantial
%% transformations.  Figure~\ref{ex:syntax-limits} illustrates this using an
%% example: Both loops in the code implement the same behaviour.  In order to
%% recognise this and apply {\em Substitute Algorithm}, pattern-based
%% approaches need explicit patterns for vastly different syntaxes implementing
%% the same semantics, which is infeasible for practical applications.

%% \todo{I've removed the reference to Move Field as it's a bit controversial given that
%% Steinmann uses it to explain that purely syntactic approaches are not enough.}

Notably, the limitations of syntax-driven refactorings have been observed
in several works, resulting in an emerging trend to incorporate more {\em semantic}
information into refactoring decisions, such as Abstract Syntax Tree (AST) 
type information,
further preventing compilation errors and behaviour changes
\cite{Steimann2011,Steimann2012Pilgrim,Steimann2011KollePilgrim}.
%% use constraints involving the abstract syntax tree type 
%% information in order to handle the {\em Move Field} (when a field
%% is used by another class more than the class on which it is defined)
%% of syntax-driven, e.g.  
%% An example 
%% of a refactoring that is well handled by the syntax-driven approach 
%% implemented using constraints over the program's abstract syntax tree
%% There is an emerging trend to incorporate more {\em semantic}
%% information into refactoring decisions, such as AST type information,
%% further preventing compilation errors and behaviour changes
%% \cite{Steimann2011}.


In this paper, we take a step further in this direction by proposing a
fully semantic refactoring approach.  There is a very broad space of
methods that are able to reason about program semantics.  The desire
to perform refactorings safely suggests the use of techniques that
overapproximate program behaviours.  As~one possible embodiment of
semantics-driven refactoring, we leverage software verification
technologies with the goal of reliably automating refactoring
decisions based on program semantics, as in the case of the {\em
  Substitute Algorithm} refactoring. Our research hypothesis is that
semantics-driven refactorings are more precise and can handle more
complex code scenarios in comparison with syntax-driven refactorings.



\paragraph{Demonstrator: Deprecated}

We use a particular refactoring as demonstrator for our idea.

\paragraph{Goal of the paper} 
%% In this paper, we are interested in refactoring Java code handing
%% collections through external iteration to use streams. Our refactoring
%% procedure is based on the program semantics and makes use of program
%% synthesis.

\paragraph{Contributions:}

\begin{itemize}

\item Use a semantics based approach to eliminate uses of deprecated APIs.

\item Introduce the concept of refactoring context. The refactoring context is different from the
  actual code to be refactored.
  We generally start
  with no context and gradually increase the amount of context we take into consideration
  until we find a desirable refactoring.
  
\item Encode the refactoring problem as a program verification problem that can be solved
  by fuzzing.
  
\end{itemize}  

%
%% \begin{itemize}
%% %
%% \item We present a program synthesis based refactoring procedure for Java
%% code that handles collections through external loop iteration.
%% %
%% \item We have implemented our refactoring method in the tool \tool. Our
%% experimental results support our conjecture that semantics-driven
%% refactorings are more precise and can handle more complex code scenarios
%% than syntax-driven refactorings.
%% %
%% \end{itemize}



\section{Overview of our approach} \label{sec:approach}

Our objective is to refactor 

The synthesis process for refactoring deprecated instances (e.g. fields, methods, classes) proceeds as follows:
\begin{enumerate}
\item {\bf Seed the synthesis library}: In this phase, we build our components library for the actual synthesis process using the sources described below.  
\item {\bf Encode the synthesis problem as a verification problem} that can be solved by fuzzing.
\end{enumerate}
  
Sources for seeding:
\begin{itemize}
\item Comments in the library.
\item Unit tests for the updated library (as opposed to the comments in libraries,
  the unit tests are syntactically valid).
\item Other github projects that have already been ported.
\end{itemize}  


\subsection{Problem encoding}\label{sec:encoding}
We encode the refactoring problem as a verification problem that can be solved by
fuzzing. We next describe the verification and synthesis phases in a CEGIS module,
where \texttt{Orig} and \texttt{Refactor} denote the original and refactored code,
respectively.

\paragraph{The verification phase} We construct the following program, where the input
\texttt{x} is used to test that the original and refactored programs return
the same output:

\begin{lstlisting}[mathescape=true,showstringspaces=false]
Verify(x) {

assert(Orig(x) == Refactor(x));

}
\end{lstlisting}

The fuzzer will find an input for which the original and refactored code
produce a different result.

\paragraph{The synthesis phase} We construct the following program, which
takes the refactored code \texttt{Refactor} as input.

\begin{lstlisting}[mathescape=true,showstringspaces=false]
Synthesise (Refactor) {

if (Orig(x1)=Refactor(x1) && ... && Orig(xn)=Refactor(xn))
   assert(false);
}
\end{lstlisting}

The fuzzer will find an input \texttt{Refactor} that fails the assertion,
meaning that it returns the same output as \texttt{Orig}
on the finite set of inputs \texttt{$x_1, \cdots, x_n$}.


%% - Performing the synthesis. Complex bit: synthesising method invocations for methods that contain loops.

%% \begin{enumerate}
%% \item We seed the templates/snippets for the synthesis library using the following two strategies:
%%   \begin{itemize}
%% \item Use deprecated block messages.
%% \item Use the unit tests for the new library. As opposed to the comments in libraries,
%%   the unit tests are syntactically valid.
%%   \end{itemize}

%% \item Use the seeded templates to generate the refactoring.
%%   \begin{itemize}
%% \item Look at checking equivalence in the presence of loops.
%%   \end{itemize}
%%   \end{enumerate}

\subsection{Checking program equivalence for a certain input}

\paragraph{Checking heap equivalence}

In Section~\ref{sec:encoding}, we simplified the property checked by
the fuzzer to \texttt{Orig(x)=Refactor(x)}, which essentially tests
whether, for input \texttt{x}, the outputs of the original and
refactored programs are the same.  However, any realistic Java program
makes modifications to the heap. Consequently, we need to check that
heap is being handled in the same manner in both programs.

Given the potential use of aliases, we will use a Hash table to
associate a symbolic id to each pointer variable such that two
variable share the same id if and only if they are aliases.
%% Then, we can check whether the set of aliases is preserved in the
%% refactored program. For instance, if \texttt{obj1} and \texttt{obj2}
%% are aliases in the original program, then they must be aliases also
%% in the second program.
\todo{check exactly what this is used for.}

\subsubsection{Loaded classes}

One of the potential issues when checking equivalence of two program
is the treatment of loaded classes. In particular, one program might
load more classes than the other, but still have the exact same effect
as the other one. This can happen for instance if one program calls a
helper method from a class that needs to be loaded, whereas the other
program inlines the same method, thus avoiding loading the class.  Our
assumption is that loadeding and initialization don't perform any
changes outside of the observable state already checked by the
existing equivalence check.  While this assumption is generally safe
to make, there might be some cases when this is not the case, e.g. the
class initializer might print something on the screen. For now, we
don't handle these cases.

%% A class is initialized when a symbol in the class is first used. When
%% a class is loaded it is not initialized.

\todo{give example}

\subsection{Research questions}

What if the new method doesn't maintain observational equivalence?
TODO: Check existing projects to see how many do preserve equivalence.
For the rest, is there a notion of partial equivalence that ensures the equivalence
for the new projects minus the differences introduced by the new function.

How easy is for the fuzzer to handle such structured inputs as we need in the
synthesis phase?


\subsection{For the experimental evaluation}
\begin{itemize}
\item Pick libraries that have deprecated instances.
\item Pick users of these libraries (i.e. open source projects that are actively maintained).
\item Make pull requests and see how many get merged.
\end{itemize}  

\section{Context-sensitive vs. context-insensitive refactoring}

\paragraph{Context-insensitive refactoring}
We consider different refactoring options. The first and most simple
one is the so called ``context-insensitive'' refactoring, where we
find a refactoring for the deprecated code that works irrespective of
context. This means that, once such a refactoring is found, it can be
used to replace all the uses of the deprecated code, e.g. a deprecated
method invocation will be replaced with the same refactored code
irrespective of the actual parameters or rest of the state at the call
site.

For instance, if we want to refactor the call to method
\texttt{toRefactor} in the code below, then we search for a refactoring that is
observationally equivalent to the given method irrespective of the
context of the call.

\begin{lstlisting}[mathescape=true]
class NoContext {
  int x;

  public double toRefactor(int y, int z) {
    return (double) x + y + z;
  }
}
\end{lstlisting}
 
This means that the state that we can vary in order to check for
observational equivalence includes \texttt{x, y} and \texttt{z}.

\paragraph{Context-sensitive refactoring}
The refactoring context consists of the context where the code to be refactored
is placed. For instance, in the snippet below, if we are trying to refactor the call to the deprecated
function \texttt{f}, then the context gives us the valuation $[\texttt{x}\mapsto 1]$ for the function's argument. 

\begin{lstlisting}[mathescape=true]
  int x = 1;
  A a = new A();
  
  a.f(1)
\end{lstlisting}
    
We now look at the possibility of considering the refactoring context
when searching for a refactoring. This gives us the option of starting
with very little context and gradually increasing it until finding a
refactoring. \todo{Adjust this based on what the experiments do.}

The other option is to find a refactoring that takes into consideration
the calling context. For instance, in the code below we have two
calls to \texttt{toRefactor}, one in method \texttt{foo} and one in
method \texttt{bar}. When considering the calling context, we have to
find two different refactorings, one for each of the method calls.
For the call in \texttt{foo}, the state that we need to consider for observational
equivalence contains \texttt{ctx} and \texttt{a}, whereas for the call in method
\texttt{bar} it contains \texttt{ctx, a, b, c, d} and \texttt{e}.

\begin{lstlisting}[mathescape=true]
  class Context1 {
    NoContext ctx = new NoContext();

    public double foo(int a) {
      return ctx.toRefactor(a, a);
    }
  }

 

  class Context2 {
    NoContext ctx = new NoContext();

    int d;
    int e;

    public double bar(int a, int b, int c) {
      return ctx.toRefactor(a, b + c + d + e);
    }
  }
\end{lstlisting}


%% Fuzzing:

%% someNoContext.x = x_*;

%% someNoContext.toRefactor(y_*, z_*);

%% Possible follow-up:
%% - if we have a sizeable codebase, the user might have to do 1,2 deprecated
%% refactorings and then the synthesisers can use the PRs done by the user
%% to seed the synthesiser and do the other thousands of refactorings
%% automatically.

\section{Implementation}

\subsection{Instrumentation}

The goal of our instrumentation is to execute a program candidate in
the same context as the original, deprecated method. The original
method can affect the program state by virtue of its return value,
field assignments in the case of instance methods, and static field
assignments. In order to account for these effects, our
instrumentation passes a reference to the original method's object
instance, if present, as an argument to the program candidate.
An example of this is presented in
Fig.~\ref{ex:side-effects-instrumentation}.

\begin{figure}
\begin{lstlisting}[mathescape=true,showstringspaces=false]
// Redundant snippet:
Date today;
Date tomorrow;
  
void init(int year, int month, int day) {
  // Original:
  this.setDates(year, month, day);

  // Instrumented:
  Program.of(0xabcd).execute(year, month, day, this);
}
  
@Deprecated
private void setDates(int year, int month, int day) {
  final Calendar calendar = Calendar.getInstance();
  calendar.set(year, month, day);
  this.today = calendar.getTime();
  calendar.add(Calendar.DAY_OF_MONTH, 1);
  this.tomorrow = calendar.getTime();
}
\end{lstlisting}
\caption{Deprecated side-effects instrumentation.}
\label{ex:side-effects-instrumentation}
\end{figure}

Fig.~\ref{ex:date-instrumentation} illustrates an example where no
such relevant instance object exists and thus only the original
arguments are passed as arguments into the program candidate.

\begin{figure}
\begin{lstlisting}[mathescape=true,showstringspaces=false]
// Redundant snippet:
static void main(String[] args) {
  // Original:
  Date date = new Date(120, 12, 30);

  // Instrumented:
  Date date = Program.of(0xabcd).execute(120, 12, 30);
}
\end{lstlisting}
\caption{Deprecated `Date' instrumentation.}
\label{ex:date-instrumentation}
\end{figure}


\section{Related Work}

Related paper: in \cite{DBLP:conf/paste/Perkins05}, the authors
replace calls to deprecated methods by their bodies.

Often, genetic improvement
\cite{DBLP:journals/dagstuhl-reports/PetkeGFL18} is used for the
purpose of code refactoring. Due to the manner in which such works measure
the fitness of potential refactorings, genetic improvement cannot vary
the refactoring context.  Conversely, our technique allows varying this
context until a refactoring is being found.

\paragraph{Program refactoring}

Cheung et al.~describe a system that automatically transforms fragments of
application logic into SQL queries~\cite{DBLP:conf/pldi/CheungSM13}. 
Moreover, similar to our approach, the authors rely on synthesis technology
to generate invariants and postconditions that validate their
transformations (a~similar approach is presented
in~\cite{DBLP:conf/cc/IuCZ10}).  The main difference (besides the actual
goal of the work, which is different from ours) to our work is that the
lists they operate on are immutable and do not support operations such as
remove.  Capturing the potential side effects caused by such operations is
one of our work's main challenges.

Syntax-driven refactoring base program transformation decisions
on observations on the program's syntax tree.  Visser
presents a purely syntax-driven framework~\cite{stratego}.  The
presented method is intended to be configurable for specific
refactoring tasks, but cannot provide guarantees about semantics
preservation.  The same holds for~\cite{txl} by Cordy et al.,
\cite{sawin} by Sawin et al., \cite{bae} by Bae et al.~and
\cite{chris} by Christopoulou et al.  In contrast to these approaches,
our procedure constructs an equivalence proof before transforming the
program. In \cite{conf/sigsoft/GyoriFDL13}, Gyori et al. present a
similar refactoring to ours but performed in a syntax-driven manner.
%
Steimann et al.~present Constraint-Based Refactoring in \cite{Steimann2011},
\cite{Steimann2012Pilgrim} and \cite{Steimann2011KollePilgrim}. Their approach
generates explicit constraints over the program's abstract syntax tree to
prevent compilation errors or behaviour changes by automated refactorings.
% This gives rise to a flexible framework of customisable refactorings,
% implementable through a refactoring constraint specification language
% (cf. \cite{Steimann2011KollePilgrim}).
The approach is limited by the information
a program's AST provides and thus favours conservative implementations of
syntax-focused refactorings such as \emph{Pull Up Field}.
%
Fuhrer et al.~implement a type constraint system to introduce missing type
parameters in uses of generic classes (cf. \cite{DBLP:conf/ecoop/FuhrerTKDK05})
and to introduce generic type parameters into classes which do not provide
a generic interfaces despite being used in multiple type contexts
(cf. \cite{DBLP:conf/icse/KiezunETF07}).
%
% Raychev et al.~present a semi-automatic approach where users perform
% incomplete refactorings manually and then employ a constraint solver
% to find a sequence of default refactorings such as move or rename
% which include the users' changes. The engine is limited to syntactic
% matching with the users' partial changes and does not consider program
% semantics~\cite{DBLP:conf/oopsla/RaychevSSV13}.
%
% Weissgerber and Diehl rely on meta information to classify changes
% between software versions as refactorings~\cite{weiss}.  The technique
% aims to identify past refactorings performed by programmers, but is
% not a decision procedure for automated refactorings.
%
O'Keffe and Cinn{\'{e}}ide present search-based
refactoring~\cite{search1, search2}, which is similar to syntax-driven
refactoring.  They rephrase refactoring as an optimisation problem,
using code metrics as fitness measure.  As such, the method optimises
syntactical constraints and does not take program semantics into
account.
%
% Bavota et al. implement refactoring decisions in \cite{Bavota:2011:IEC}
% using semantic information limited to identifiers and comments,
% which may differ from the actual semantics (e.g. due to bugs).
%
Kataoka et al. interpret program semantics to apply refactorings
\cite{Kataoka:2001:ASP:846228.848644}, but use dynamic test execution
rather than formal verification, and hence their transformation lacks
soundness guarantees.
%
Franklin et al. implement a pattern-based refactoring approach
transforming statements to stream queries~\cite{Gyori:2013:CGI:2491411.2491461}.
Their tool LambdaFicator~\cite{DBLP:conf/icse/FranklinGLD04} is available as a
NetBeans branch. We compared \tool against it in our experimental evaluation
in Sec.~\ref{experiments-results}.
%
%% \paragraph{Heap Logics}
%
%% While many decidable heap logics have been developed recently, none are
%% expressive enough to capture operations allowed by the Java Collection
%% interface, operations allowed by the Java Stream interface as well as
%% equality between collections (for lists this implies that we must be able to
%% reason about both content of lists and the order of
%% elements)~\cite{DBLP:conf/cav/ItzhakyBINS13, DBLP:conf/cav/PiskacWZ13,
%% DBLP:conf/esop/BrainDKS14, DBLP:conf/popl/MadhusudanPQ11,
%% DBLP:conf/atva/BouajjaniDES12, DBLP:conf/lpar/DavidKL15}.  On the other
%% hand, very expressive transitive closure
%% logics~\cite{DBLP:conf/csl/ImmermanRRSY04} are not concise and easily
%% translatable to stream code.

\paragraph{Program synthesis}

An approach to program synthesis very similar to ours is Syntax Guided
Synthesis (SyGuS)~\cite{sygus}.  SyGuS synthesisers supplement the logical
specification with a syntactic template that constrains the space of allowed
implementations.  Thus, each semantic specification is accompanied by a
syntactic specification in the form of a grammar.  Other second-order
solvers are introduced in~\cite{DBLP:conf/pldi/GrebenshchikovLPR12,
DBLP:conf/cav/BeyenePR13}.  As opposed to ours, these focus on
Horn clauses.

\section{Conclusion}

We conjecture that refactorings driven by the semantics of programs have
broader applicability and are able to address more complex refactoring
schemata in comparison to conventional syntax-driven refactorings, thereby
increasing the benefits of automated refactoring.  The space of possible
semantic refactoring methods is enormous; as an instance, we have presented
a method for refactoring iteration over Java collection classes based on
program synthesis methods.  Our experiments indicate that refactoring using
this specific instance is feasible, sound and sufficiently performant. 
Future research must broaden the evidence for our general hypothesis by
considering other programming languages, further, ideally more complex
refactoring schemata, and other semantics-based analysis techniques.

%\acks

% We recommend abbrvnat bibliography style.

\bibliographystyle{abbrv}
\bibliography{document}{}



\end{document}

%                       Revision History
%                       -------- -------
%  Date         Person  Ver.    Change
%  ----         ------  ----    ------

%  2013.06.29   TU      0.1--4  comments on permission/copyright notices

