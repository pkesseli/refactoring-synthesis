%-----------------------------------------------------------------------------
%               Template for OOPSLA
%               based on:
%               Template for sigplanconf LaTeX Class
%
% Name:         sigplanconf-template.tex
%
% Purpose:      A template for sigplanconf.cls, which is a LaTeX 2e class
%               file for SIGPLAN conference proceedings.
%
% Guide:        Refer to "Author's Guide to the ACM SIGPLAN Class,"
%               sigplanconf-guide.pdf
%
% Author:       Paul C. Anagnostopoulos
%               Windfall Software
%               978 371-2316
%               paul@windfall.com
%
% Created:      15 February 2005
%
%-----------------------------------------------------------------------------
\documentclass[sigconf,review,anonymous]{acmart}
\acmConference[ESEC/FSE 2022]{The 30th ACM Joint European Software Engineering Conference and Symposium on the Foundations of Software Engineering}{14 - 18 November, 2022}{Singapore}
%\documentclass[runningheads,a4paper]{llncs}
%\documentclass[sigconf,authordraft]{acmart}
%\acmConference[ESEC/FSE 2017]{11th Joint Meeting of the European Software Engineering Conference and the ACM SIGSOFT Symposium on the Foundations of Software Engineering}{4--8 September, 2017}{Paderborn, Germany}
%\documentclass[10pt,numbers]{sigplanconf}
%\usepackage[1stsubmission]{oopsla2016}
%\usepackage[2ndsubmission]{oopsla2016}

%\usepackage[scaled]{helvet} % see www.ctan.org/get/macros/latex/required/psnfss/psnfss2e.pdf

%% \BibTeX command to typeset BibTeX logo in the docs
\AtBeginDocument{%
  \providecommand\BibTeX{{%
    \normalfont B\kern-0.5em{\scshape i\kern-0.25em b}\kern-0.8em\TeX}}}


\usepackage{microtype}
\usepackage{url}                  % format URLs
\usepackage{listings}          % format code
\lstset{
  mathescape, 
  language={Java},
  basicstyle=\footnotesize
}
\usepackage{enumitem}      % adjust spacing in enums
%\usepackage[colorlinks=true,allcolors=blue,breaklinks,draft=false]{hyperref}   % hyperlinks, including DOIs and URLs in bibliography
% known bug: http://tex.stackexchange.com/questions/1522/pdfendlink-ended-up-in-different-nesting-level-than-pdfstartlink

\usepackage{graphicx}
\usepackage{float,subfig}
\usepackage{xspace,framed}
\usepackage{colortbl}
\usepackage{calc}
\usepackage[ruled,vlined]{algorithm2e}

%\usepackage{algorithm}
\usepackage{amsfonts}
\usepackage{amsmath}
\usepackage{multicol}
\usepackage{multirow}
\usepackage{tikz}
\usepackage[justification=centering]{caption}
\usepackage{stmaryrd}
\usetikzlibrary{positioning, arrows, automata, shapes}
\usepackage{hhline}
\usepackage{pifont}
%\usepackage{cite}
\usepackage{pdflscape} % Experiments table is landscape
\usepackage{longtable}
\usepackage{afterpage}
\usepackage{wasysym}

\usepackage[justification=centering]{caption}

\newcommand{\doi}[1]{}{}%{doi:~\href{http://dx.doi.org/#1}{\Hurl{#1}}}   % print a hyperlinked DOI

\newcommand{\cbmc}{CBMC}
\newcommand{\sat}{SAT}

% Theorems and such

%\newtheorem{definition}{Definition}
%\newtheorem{lemma}{Lemma}
%\newtheorem{example}{Example}
%\newtheorem{theorem}{Theorem}

\newcommand{\defeq}{\ensuremath{\stackrel{\mathrm{def}}{=}}}

% \algrenewcommand\algorithmicrequire{\textbf{Precondition:}}
% \algrenewcommand\algorithmicensure{\textbf{Postcondition:}}


\newcommand{\univ}{\ensuremath{\mathcal{P}}\xspace}
\newcommand{\ex}{\ensuremath{\mathcal{E}}\xspace}
\newcommand{\hlen}{\ensuremath{\mathit{len}}\xspace}
\newcommand{\hskips}{\ensuremath{\mathit{skip}}\xspace}
\newcommand{\hlimit}{\ensuremath{\mathit{limit}}\xspace}
\newcommand{\hmax}{\ensuremath{\mathit{max}}\xspace}
\newcommand{\hmin}{\ensuremath{\mathit{min}}\xspace}
\newcommand{\hforeach}{\ensuremath{\mathit{foreach}}\xspace}
\newcommand{\hmap}{\ensuremath{\mathit{map}}\xspace}
\newcommand{\hsiz}{\ensuremath{\mathit{size}}\xspace}
\newcommand{\hval}{\ensuremath{\mathit{val}}\xspace}
\newcommand{\hnodes}{\ensuremath{\mathit{nodes}}\xspace}
\newcommand{\hedges}{\ensuremath{\mathit{edges}}\xspace}

% Logic operators

\newcommand{\halias}{\ensuremath{\mathit{alias}}\xspace}
\newcommand{\heapsize}{\ensuremath{\mathit{size}}\xspace}
\newcommand{\hispath}{\ensuremath{\mathit{isPath}}\xspace}
\newcommand{\hfilter}{\ensuremath{\mathit{filter}}\xspace}
\newcommand{\hnew}{\ensuremath{\mathit{new}}\xspace}
\newcommand{\hassign}{\ensuremath{\mathit{assign}}\xspace}
\newcommand{\halloc}{\ensuremath{\mathit{alloc}}\xspace}
\newcommand{\hset}{\ensuremath{\mathit{set}}\xspace}
\newcommand{\hget}{\ensuremath{\mathit{get}}\xspace}
\newcommand{\hremove}{\ensuremath{\mathit{remove}}\xspace}
\newcommand{\hadd}{\ensuremath{\mathit{add}}\xspace}

% Logic extension operators

\newcommand{\iterators}{\ensuremath{\mathit{I}}\xspace}
\newcommand{\lists}{\ensuremath{\mathit{L}}\xspace}

\newcommand{\id}{\ensuremath{\mathrm{id}}\xspace}
\newcommand{\hsubdivide}{\ensuremath{\mathit{subdivide}}\xspace}
\newcommand{\hsmoothen}{\ensuremath{\mathit{smoothen}}\xspace}
\newcommand{\hfresh}{\ensuremath{\mathit{fresh}()}\xspace}
\newcommand{\hexists}{\ensuremath{\mathit{exists}}\xspace}
\newcommand{\hforall}{\ensuremath{\mathit{forall}}\xspace}
\newcommand{\hsorted}{\ensuremath{\mathit{sorted}}\xspace}
\newcommand{\hls}[2]{#1 {\rightarrow^*} #2}
\newcommand{\hnull}{\ensuremath{\mathbf{null}}}
\newcommand{\sem}[1]{\ensuremath{\llbracket{#1}\rrbracket}}
% \newcommand{\xmark}{\ding{55}}

\newcommand*\Let[2]{\State #1 $\gets$ #2}

\newcommand{\bv}[2]{\mathcal{BV}(#1, #2)}

\newcommand{\todo}[1]{{\color{red} TODO: {#1}}} 

\iffalse
\lstset{emph={
  assume, assert}, emphstyle=\bfseries}
\fi


% Logic things

\newcommand{\true}{\textit{true}\xspace}
\newcommand{\false}{\textit{false}\xspace}
\newcommand{\unknown}{\ensuremath{\mathsf{?}}\xspace}

% Heap variables

\newcommand{\numNodes}{\ensuremath{N}\xspace}
\newcommand{\numPointers}{\ensuremath{|PV|}\xspace}
\newcommand{\numPositional}{\ensuremath{NPos}\xspace}
\newcommand{\numPredicates}{\ensuremath{\mathsf{NPred}}\xspace}
\newcommand{\allocatedNodes}{\ensuremath{\mathsf{NAlloc}}\xspace}

% Java collections section

\newcommand{\listToPtr}[1]{\ensuremath{#1^{\mathcal{P}}}\xspace}
\newcommand{\preHeap}{\ensuremath{h}\xspace}
\newcommand{\postHeap}{\ensuremath{h'}\xspace}
\newcommand{\freshP}{\ensuremath{p}\xspace}

\newcommand{\hsucc}{\ensuremath{succ}\xspace}
\newcommand{\atindex}{\ensuremath{@}\xspace}
\newcommand{\swapNext}{\ensuremath{swap}\xspace}
\newcommand{\updateI}{\ensuremath{upI}\xspace}
\newcommand{\updateL}{\ensuremath{upL}\xspace}

\newcommand{\hprev}{\ensuremath{\mathit{prev}}\xspace}

% Java operators

\newcommand{\java}[1]{\ensuremath{\mathtt{#1}}\xspace}
\newcommand{\jadd}{\ensuremath{\java{add}}\xspace}
\newcommand{\jlist}{\ensuremath{\java{ArrayList}}\xspace}
\newcommand{\jnew}{\ensuremath{\java{new}}\xspace}
\newcommand{\jset}{\ensuremath{\java{set}}\xspace}
\newcommand{\jcontains}{\ensuremath{\java{contains}}\xspace}
\newcommand{\jcontainsall}{\ensuremath{\java{containsAll}}\xspace}
\newcommand{\jequals}{\ensuremath{\java{equals}}\xspace}
\newcommand{\jindexof}{\ensuremath{\java{indexOf}}\xspace}
\newcommand{\jlastindexof}{\ensuremath{\java{lastIndexOf}}\xspace}
\newcommand{\jremove}{\ensuremath{\java{remove}}\xspace}
\newcommand{\jremoveall}{\ensuremath{\java{removeAll}}\xspace}
\newcommand{\jretainall}{\ensuremath{\java{retainAll}}\xspace}
\newcommand{\jaddall}{\ensuremath{\java{addAll}}\xspace}
\newcommand{\jtoarray}{\ensuremath{\java{toArray}}\xspace}
\newcommand{\jclear}{\ensuremath{\java{clear}}\xspace}
\newcommand{\jget}{\ensuremath{\java{get}}\xspace}
\newcommand{\jhashcode}{\ensuremath{\java{hashCode}}\xspace}
\newcommand{\jisempty}{\ensuremath{\java{isEmpty}}\xspace}
\newcommand{\jiterator}{\ensuremath{\java{iterator}}\xspace}
\newcommand{\jlistiterator}{\ensuremath{\java{listIterator}}\xspace}
\newcommand{\jsize}{\ensuremath{\java{size}}\xspace}
\newcommand{\jsublist}{\ensuremath{\java{subList}}\xspace}
\newcommand{\jhasnext}{\ensuremath{\java{hasNext}}\xspace}
\newcommand{\jnext}{\ensuremath{\java{next}}\xspace}


% Helper formulas



% implementation data-structure variables

\newcommand{\code}[1]{\lstinline|#1|} % TODO: This is barely visible!
\newcommand*{\codenospace}[1]{\lstinline|#1|\kern-1ex}%
\newcommand{\ptr}{\lstinline|ptr|}
\newcommand{\succs}{\lstinline|succ|}
\newcommand{\prev}{\lstinline|prev|}
\newcommand{\data}{\lstinline|data|}
\newcommand{\val}{\lstinline|val|}
\newcommand{\dist}{\lstinline|dist|}
\newcommand{\pred}{\lstinline|pred|}
\newcommand{\sorted}{\lstinline|sorted|}
\newcommand{\upperBound}{\lstinline|ub|}
\newcommand{\lowerBound}{\lstinline|lb|}

% Name of the logic

\newcommand{\logic}{JST\xspace}
\newcommand{\tool}{Kayak\xspace}
\newcommand{\badlogic}{JCA+sharing\xspace}

\newcommand{\equivState}{\ensuremath{\mathit{equivState}}\xspace}
\newcommand{\heapEquiv}{\ensuremath{\mathit{heapEquiv}}\xspace}

% Results

\newcommand{\cmark}{\ding{51}}%
\newcommand{\xmark}{\ding{55}}%
\newcommand{\bench}[1]{{\footnotesize\textsf{#1}}\xspace}

%%% Local Variables:
%%% mode: latex
%%% TeX-master: "paper"
%%% End:


\tikzset{
    %Define style for boxes
    box/.style={
           rectangle,
           rounded corners,
           draw=black, very thick,
           fill=yellow!20,
           text width=10em,
           minimum height=2em,
           text centered},
 % 
  futurebox/.style={
           rectangle,
           rounded corners,
           draw=black, very thick,
           fill=blue!10,
           text width=10em,
           minimum height=2em,
           text centered},
 % 
  longfuturebox/.style={
           rectangle,
           rounded corners,
           draw=black, very thick,
           fill=purple!20,
           text width=10em,
           minimum height=2em,
           text centered},
%
    % Define arrow style
    arrow/.style={
           ->,
           thick,
           shorten <=2pt,
           shorten >=2pt,}
}


%\setlength{\belowcaptionskip}{-10pt}
\setlength{\textfloatsep}{1em}
\setlength{\dbltextfloatsep}{1em}
\setlength{\abovecaptionskip}{0.1em}
% \floatsep: space left between floats (12.0pt plus 2.0pt minus 2.0pt).
% \textfloatsep: space between last top float or first bottom float and the text (20.0pt plus 2.0pt minus 4.0pt).
% \intextsep : space left on top and bottom of an in-text float (12.0pt plus 2.0pt minus 2.0pt).
% \dbltextfloatsep is \textfloatsep for 2 column output (20.0pt plus 2.0pt minus 4.0pt).
% \dblfloatsep is \floatsep for 2 column output (12.0pt plus 2.0pt minus 2.0pt).
% \abovecaptionskip: space above caption (10.0pt).
% \belowcaptionskip: space below caption (0.0pt).

\allowdisplaybreaks

%%%%%%%%%%%%%%%%%%%%%%%%%%%%%%%%%%%%%
%% uncomment for extended version 
\newcommand*{\extended}{}
%%%%%%%%%%%%%%%%%%%%%%%%%%%%%%%%%%%%%


% The following oopsla2016 options are available:
%
% 1stsubmission   For the initial submission
% 2ndsubmission   For the 2nd submission
% final           For camera-ready

\begin{document}

%\copyrightdata{978-1-nnnn-nnnn-n/yy/mm} 
%\doi{nnnnnnn.nnnnnnn}

% Uncomment one of the following two, if you are not going for the 
% traditional copyright transfer agreement.

%\exclusivelicense                % ACM gets exclusive license to publish, 
                                  % you retain copyright

%\permissiontopublish             % ACM gets nonexclusive license to publish
                                  % (paid open-access papers, 
                                  % short abstracts)

\title{Refactoring of Java deprecated guided by types and code hints}
%\subtitle{Subtitle Text, if any}
% double-blind submission
% single-blind only for Technical Research: http://icse2017.gatech.edu/technical-research-cfp
 % \authorinfo{Cristina David}
 %            {University of Oxford}
 %            {cristina.david@cs.ox.ac.uk}
 % \authorinfo{Pascal Kesseli}
 %            {University of Oxford}
 %            {pascal.kesseli@cs.ox.ac.uk}
 % \authorinfo{Daniel Kroening}
 %            {University of Oxford}
 %            {kroening@cs.ox.ac.uk}

\author{Cristina David}
\email{cristina.david@bristol.ac.uk}
\orcid{0000-0002-9106-934X}
\affiliation{
  \institution{University of Bristol}  
  \city{Bristol}  
  \country{UK}
  \postcode{BS8 1TH}
}
\author{Pascal Kesseli}
\affiliation{
  \institution{Diffblue}  
  \country{UK}
}
\author{Daniel Kroening}
\affiliation{
  \institution{Amazon}  
  \country{UK}
}


%% \author{
%% \mbox{Cristina David}\inst{1} \and
%% Pascal Kesseli\inst{2} \and
%% Daniel Kroening\inst{2}}

%% \institute{University of Bristol, UK 
%% \and
%%   University of Oxford, UK}

%% \author{Cristina David}
%% \affiliation{University of Oxford}
%% \email{cristina.david@cs.ox.ac.uk}

%% \author{Pascal Kesseli}
%% \affiliation{University of Oxford}
%% \email{pascal.kesseli@cs.ox.ac.uk}

%% \author{Daniel Kroening}
%% \affiliation{University of Oxford}
%% \email{kroening@cs.ox.ac.uk}

\begin{abstract}
%
  Refactorings are structured changes to existing software that leave its
  externally observable behaviour unchanged.  Their intent is to improve
  readability, performance or other non-behavioural properties. 
  State-of-the-art automatic refactoring tools are {\em syntax}-driven and,
  therefore, overly conservative.  In this paper we explore {\em
  semantics}-driven refactoring, which enables much more sophisticated
  refactoring schemata.  As~an exemplar of this broader idea, we present
  an automatic refactoring tool that replaces uses of deprecated legacy APIs/libraries
  with the latest ones.
  Our refactoring procedure performs
  semantic reasoning and search in the space of possible refactorings using
  automated program synthesis.  Our experimental results support the
  conjecture that semantics-driven refactorings are more precise and are
  able to rewrite more complex code scenarios when compared to syntax-driven
  refactorings.
%
\end{abstract}

\maketitle

%\keywords{program refactoring, program synthesis, program verification}

%\category{CR-number}{subcategory}{third-level}

% general terms are not compulsory anymore, 
% you may leave them out
%\terms
%term1, term2

%% \keywords
%% Maintenance and reuse



\section{Introduction}\label{sec:intro}

As a project evolves, there are certain fields, methods or classes
that the developers are discouraged from using in the future as
they've been superseded and may cease to exist in the future.
However, removing them directly would break the backward compatibility
of the project's API.  Instead, such elements can be tagged with the
\code{@Deprecated} annotation.  %% In general, when deprecating a
%% field/method/class, the \code{@deprecated} Javadoc tag is used in the
%% comment section to inform the developer the reason of deprecation and
%% what can be used in place.


%Even in the presence of such recommendations,
The transformation of existing code such that it doesn't use deprecated APIs
%to the recommended format
is not always straightforward,
as illustrated in Fig.~\ref{ex:deprecated-method-other}. In the example, we
make use of the \code{getHours} method of the \code{Date} class,
which is deprecated.
%% with the 
%% recommendation to use \code{Calendar.get(Calendar.HOUR\_OF\_DAY)} instead.
In this situation, in order to replace the use of the deprecated method,
we must first obtain a \code{Calendar} object.
However, we can't use the \code{Calendar} constructor as it is protected,
and we must instead call \code{getInstance}.
Furthermore, in order to be able to use this \code{Calendar} object for our purpose,
we must first set its time using the existing \code{date}.
We do this by calling \code{setTime} with \code{date} as argument.
%% followed by setting the date using \code{set}
%% and returning a Date object by invoking \code{getTime}.
%% Alternatively, the \code{GregorianCalendar} class may be used with
%% similar difficulties.
Finally, we can retrieve the hour by calling
%Only after all this set up, we can finally make use of the recommendation
\code{calendar.get(Calendar.HOUR\_OF\_DAY)}.

\begin{figure}
\begin{lstlisting}[mathescape=true,showstringspaces=false]
void main(String[] args) {
  // Deprecated:
  int hour = date.getHours();
  
  // Should have been:
  final Calendar calendar = Calendar.getInstance();
  calendar.setTime(date);
  int hour = calendar.get(Calendar.HOUR_OF_DAY);
}
\end{lstlisting}
\caption{Deprecated method example.}
\label{ex:deprecated-method-other}
\end{figure}


While it is trivial for IDEs such as Eclipse or IntelliJ IDEA to
recognise the use of deprecated APIs and warn the programmer about it,
refactoring the code to a preferred API is significantly more
complex. Indeed neither Eclipse 4.15.0 nor IntelliJ IDEA 2020.2.2 make
any attempt to suggest a better alternative to the user. There is also
very limited support offered by the research community, mainly
focusing on replacing calls to deprecated methods by their
bodies~\cite{DBLP:conf/paste/Perkins05}. 

Consequently, the majority of
such refactorings are done manually, which is a costly,
time-intensive, and not least error-prone process.  This situation is
also a real issue for legacy code that makes use of deprecated APIs
that need to be replaced with latest APIs/libraries.

The main cause of difficulty when replacing deprecated instances lies
with the fact that refactorings should not change the functionality of
the code, i.e., they must preserve observational equivalence.
Ensuring observational equivalence is far from trivial.  For instance,
inlining calls to deprecated methods can break equivalence if the
original function was synchronised.  While it is not possible for two
invocations  on the same object of the synchronized original method to
interleave, this is not guaranteed after inlining the method's body.
Such an exampe is given in Figure~\ref{}.
\todo{add example here.}

%% Another example of a refactoring that unintuitively breaks
%% observational equivalence is given in Figure~\ref{}.
%% \todo{add example here.}

Consequently, refactorings that rely solely on a program's syntax
and do not consider its semantics end up
being overly conservative, mostly addressing structural changes
to the program.  For illustration, when referring back to the example in
Figure~\ref{ex:deprecated-method-other}, one must understand the code's semantics in order to conclude that the
refactoring maintains the same functionality.
%
%% such refactorings are based on semantics rather
%% than syntax. For illustration, let's refer back to the example in
%% Figure~\ref{ex:deprecated-method}.  A {\em syntax-driven refactoring}
%% addresses structural changes to the program requiring only limited
%% information about a program's semantics, whereas a {\em
%%   semantics-driven refactoring} requires detailed understanding of the
%% program semantics in order to be applied soundly.  Clearly, the
%% refactoring of the call to the deprecated \code{Date} constructor
%% falls into the latter category. 
%
%\subsection{From the previous introduction:}
%% Refactorings are structured changes to existing software which leave its
%% externally observable behaviour unchanged.  They improve non-functional
%% properties of the program code, such as testability, maintainability and
%% extensibility while retaining the semantics of the program.  Ultimately,
%% refactorings can improve the design of code, help finding bugs as well
%% as increase development speed and are therefore seen as an integral part
%% of agile software engineering processes~\cite{DBLP:conf/xpu/Kerievsky04b,
%% Fowler1999}.
%
%% However, manual refactorings are a costly, time-intensive, and not least
%% error-prone process.  This has motivated work on automating specific
%% refactorings, which promises safe application to large code bases at low
%% cost.  We differentiate in this context between {\em syntax-driven} and {\em
%% semantics-driven} refactorings.  While the former address structural changes
%% to the program requiring only limited information about a program's
%% semantics, the latter require detailed understanding of the program
%% semantics in order to be applied soundly.  An example of a refactoring that
%% requires a semantics-driven approach is {\em Substitute Algorithm}, where an
%% algorithm is replaced by a clearer, but equivalent
%% version~\cite{Fowler1999}.
%
%AST-based constraints 
%
%% A syntax-driven approach is insufficient to perform such substantial
%% transformations.  Figure~\ref{ex:syntax-limits} illustrates this using an
%% example: Both loops in the code implement the same behaviour.  In order to
%% recognise this and apply {\em Substitute Algorithm}, pattern-based
%% approaches need explicit patterns for vastly different syntaxes implementing
%% the same semantics, which is infeasible for practical applications.
%
%% \todo{I've removed the reference to Move Field as it's a bit controversial given that
%% Steinmann uses it to explain that purely syntactic approaches are not enough.}
%
Notably, the limitations of syntax-driven refactorings have been observed
in several works, resulting in an emerging trend to incorporate more {\em semantic}
information into refactoring decisions, such as Abstract Syntax Tree (AST) 
type information,
further preventing compilation errors and behaviour changes
\cite{Steimann2011,Steimann2012Pilgrim,Steimann2011KollePilgrim}.
%% use constraints involving the abstract syntax tree type 
%% information in order to handle the {\em Move Field} (when a field
%% is used by another class more than the class on which it is defined)
%% of syntax-driven, e.g.  
%% An example 
%% of a refactoring that is well handled by the syntax-driven approach 
%% implemented using constraints over the program's abstract syntax tree
%% There is an emerging trend to incorporate more {\em semantic}
%% information into refactoring decisions, such as AST type information,
%% further preventing compilation errors and behaviour changes
%% \cite{Steimann2011}.

In this paper, we take a step further in this direction by proposing a
semantics-driven refactoring technique that makes use of program synthesis.
While we focus on the refactoring of deprecated methods, the same
technique can be applied to deprecated fields and classes.
%Our goal is to design a refactoring techniques for deprecated instances that computes refactorings fast, thus allowing integration with an IDE.

%% As explained in more detail in the overview in Section~\ref{sec:overview},
%% the program synthesis technique will be guided by {\em code hints}
%% and {\em types}, as well 

Although powerful, program synthesis is very hard due to the vast search space, i.e. space of possible programs that must be considered. Thus, its success is generally relient on how efficiently it is guided towards a solution by discarding uninteresting programs as early as possible. Most commonly, the information used to guide the search is represented by input/output examples~\cite{DBLP:conf/pldi/FeserCD15}, type signatures~\cite{DBLP:conf/pldi/OseraZ15}, or full functional specifications of the expected code~\cite{DBLP:conf/ijcai/MannaW79}.
%
The vast search space is particularly a challenge in the current setting given that, in the absence of any additional guidance, the entire \code{java.lang} needs to be considered during code generation.
In order to prune the search space, we start by investigating the information available to our synthesis procedure.
%The information that we are going to use to direct
%\begin{itemize}

{\bf (1)} {\em Type information} about the code to be generated. In particular, this consists on the types of the objects to be consumed (i.e. the inputs), as well as the types of the objects that need to be produced (i.e. the outputs). As mentioned above, types are commonly used to guide  program synthesis. However, as shown by our experimental evaluation in Section~\ref{sec:experimental-results}, while useful, they are not sufficient in the current scenario. %, where the number of possible programs where considering \code{java.lang} %\cite{DBLP:conf/popl/FengM0DR17} \todo{add more citations}.

{\bf (2)} {\em Code hints} present in the Javadoc comments. In general, when deprecating a field/method/class, the \code{@deprecated} Javadoc tag is used in the
  comment section to inform the developer of the reason for deprecation and what can be used in its place. As we will show in Section~\ref{sec:overview}, these hints
  don't provide the whole refactoring, but can be used to guide the search process.

  {\bf (3)} One last information we have about the code to be generated is that it needs to be {\em semantically equivalent to the original}. %While this does provide us with the {\em semantics of the
  %code to be generated}, it's not easy to use. 
%* This is not enough to decide semantic equivalence, which, as discussed above, is one of the main difficulties in this refactoring.
  However, formally proving semantic equivalence is expensive and undecidable in the general case. %very difficult, having been the subject of extended research over the years \cite{}.
%Even when theoretically possible, the process of proving semantic equivalence is very expensive as it generally requires full understanding of the code's semantics.
%Moreover, it generally requires full understanding of a program's internal structure, which is expensive.
%This contradicts our aforementioned goal to design a technique that is fast. In a bid to obtain good performance,
%we investigated using greybox testing for checking program equivalence, rather than formally proving it.
In this work, instead of aiming for a formal proof,  we use coverage-guided property-based testing to check equivalence. %, which only requires limited instrumentation of the programs under test.
While we can't provide full correctness guarantees,
our experiments show that, when considering a large number of test inputs, the results are sound.

%Then, we choose a compromise between performance and ease of use and soundness. We use ...

%{\em Following the same principle, we design a greybox synthesis technique that is guided by code hints and types.}


%% For this purpose, we make use of component-based program synthesis, which assembles programs
%% from a library of existing components, e.g. methods provided by an API.
%% One of the challenges in component-based synthesis is the large number of components
%% when considering real-world APIs. Specifically for our use case, when generating Java code, we have
%% access to the whole of Java standard library.
%% We address this challenge by using the code hints provided in the Javadoc comments
%% (under the Javadoc \code{@code} tag) and type information to prune the component
%% library before the actual synthesis process.

%% With respect to the synthesis process, we propose using greybox
%% synthesis. Traditionally, synthesis techniques are either based
%% on whitebox or blackbox approaches.  The former are fully aware of the
%% internal structure of the program being synthesised, whereas the
%% latter only examine its functionality. In this paper, we use a greybox
%% approach, which relies on coverage-guided property-based testing. 
%% For this purpose, we provide an encoding for the synthesis problem, which
%% makes it amenable to solving via coverage-guided property-based testing.


%% In this paper, we follow a two-step procedure to address this challenge:
%% \begin{itemize}
%% \item[{\bf Step 1}] We start from the code hints provided as Javadoc comments, and use
%%   type guidance to seed a components library to be used during Step 2. 

%% \item[{\bf Step 2}] Based on the components library built during the first step, we use program synthesis to find a refactoring that is semantically equivalent to the original code.
%% \end{itemize}

%% syntax and semantics based approach based on program
%% synthesis.  Essentially, we make use of syntactic information from the
%% original code as well as the Javadoc comments to seed a components
%% library, which is then used by a program synthesis engine.
%% Notably, the synthesiser automatically generates refactorings
%% that are guaranteed to be semantically equivalent to the original
%% code.  With respect to the actual program synthesis technique, we use
%% a coverage-based, type and example guided synthesis.


%% fully semantic refactoring approach for removing deprecated elements.
%% There is a very broad space of methods that are able to reason about
%% program semantics.  The desire to perform refactorings safely suggests
%% the use of techniques that overapproximate program behaviours.  As one
%% possible embodiment of semantics-driven refactoring, we leverage
%% software verification technologies with the goal of reliably
%% automating refactoring decisions based on program semantics.
%% Our research
%% hypothesis is that semantics-driven refactorings are more precise and
%% can handle more complex code scenarios in comparison with
%% syntax-driven refactorings.

%% \todo{revise the next two paragraphs}
%% \paragraph{Goal of the paper} 
%% Our goal in this paper is to design a fully automated refactoring
%% technique for eliminating deprecated fields, methods and classes. Our
%% technique will make use of Java library comments to find components
%% for seeding the refactoring. Then, it will use program synthesis to
%% generate code that is semantically equivalent to the original but does not use
%% the deprecated elements.

%% One interesting aspect when using a fully semantics-driven approach is
%% that it allows introducing the notion of {\em refactoring context}.
%% The refactoring context is the context (of the code to be refactored)
%% taken into consideration when looking for a refactoring. Varying the
%% context enables the user to control the generality of the refactoring.
%% \todo{Are we going to cover context?}

%
%% For illustration, for the refactoring in Figure~\ref{ex:deprecated-method-other}...
%% \todo{hmm, what context do we consider for the standard library?}
%
%\todo{Should we also discuss our choice of synthesis in the intro?}

%% In this paper, we are interested in refactoring Java code handing
%% collections through external iteration to use streams. Our refactoring
%% procedure is based on the program semantics and makes use of program
%% synthesis.

\paragraph{Contributions:}

\begin{itemize}

\item We propose a semantics based approach to eliminate uses of deprecated APIs, which employs program synthesis.

\item We propose a greybox program synthesis technique that is driven by type information and code hints, and relies on coverage-guided property-based testing.
  
%\item We provide a technique for building the component library that is guided by code hints and types.
  
%\item We propose greybox program synthesis by encoding the code generation problem as a verification problem that can be solved by coverage-guided property-based testing.
  %% introduce the notion of coverage based CEGIS, which makes use of coverage-guided property-based testing in order to generate counterexamples for a type and example guided synthesiser.
  
%% \item We provide an encoding of the synthesis problem as the problem of verifying the safety of a program, which is then solved by fuzzing.

%% \item We introduce the concept of refactoring context. %% The refactoring context is different from the
%%   %% actual code to be refactored.
%%   Our refactoring technique generally starts 
%%   with no context and gradually increase the amount of context we take into consideration
%%   until a desirable refactoring is found.

\item We Implemented our technique in the tool ?? and used it to refactor the methods deprecated in jdk15.
  %refactor deprecated instances in X open source projects.
  
\end{itemize}  

%
%% \begin{itemize}
%% %
%% \item We present a program synthesis based refactoring procedure for Java
%% code that handles collections through external loop iteration.
%% %
%% \item We have implemented our refactoring method in the tool \tool. Our
%% experimental results support our conjecture that semantics-driven
%% refactorings are more precise and can handle more complex code scenarios
%% than syntax-driven refactorings.
%% %
%% \end{itemize}



\section{Overview of our approach} \label{sec:overview}

As mentioned in Section~\ref{sec:intro}, our approach employs program synthesis
in order to generate a refactoring that is semantically equivalent to the original code and
does not use deprecated APIs.
Essentially, the new code is obtained by composing methods
that are accessible from that respective location. 
Consequently, we make use of component-based program synthesis,
which weaves together components from a library
(typically methods from an API) in order to generate the desired program \cite{DBLP:conf/icse/JhaGST10,DBLP:conf/pldi/GulwaniJTV11,DBLP:conf/popl/FengM0DR17}.
%Component-based program synthesis allows us to 

%% While powerful, program synthesis is very hard due to the vast
%% search space. In particular,
For component-based synthesis, real-world
APIs can be very large. %, making synthesis infeasible.
For illustration, in our setting, indiscriminately adding all members
found in \code{java.lang} would make synthesis infeasible (see experimental results in Section~\ref{sec:experimental-results}).
%% In general program synthesis is very difficult. One of the main
%% challenges is the very large search space.
%% The objective of recent
%% program synthesis research has been to reduce the search space.
%% One such direction, component-based program synthesis allows
%% synthesis techniques to weave together components from a library
%% (typically methods from an API) in order to generate the desired program.
%% While this does improve the efficiency of program synthesis by providing
%% higher order building blocks that can be used..., real-world APIs are still
%% too big.
%
Previous works have used type information to make components-based synthesis
feasible in the presence of large component
libraries~\cite{DBLP:conf/popl/FengM0DR17}.  However, as shown by our
experimental results discussed in
Section~\ref{sec:experimental-results}, the use of types in our setting is insufficient
for obtaining a fast refactoring technique that is able to
provide hints to the developer on-the-fly.
%
%% For related work:
%% However, that worked well when the generated code
%% only had to be type sound and obey at most three test cases. However, in our case
%% we need to generate code that is observationally equivalent to the original.
%% While we will use coverage-guided testing to check observational equivalence (as discussed later),
%% we still check at least 400 inputs. 
%
Our solution is to use the code hints provided in the Javadoc comments to guide the building of the component library. For our running example in Figure~\ref{ex:deprecated-method-other}, the source code for the \code{getHours} method in class \code{Date} is accompanied by the comment in Figure~\ref{ex:code-hints}, where the \code{@code} tag suggests replacing the deprecated \code{getHours}
with \code{Calendar.get(Calendar.HOUR_OF_DAY)}.


\begin{figure}
\begin{lstlisting}[mathescape=true,showstringspaces=false]
/*
 * @return  the hour represented by this date.
 * @see     java.util.Calendar
 * @deprecated As of JDK version 1.1,
 * replaced by {@code Calendar.get(Calendar.HOUR_OF_DAY)}.
 */  
\end{lstlisting}
\caption{Code hints for the running example.}
\label{ex:code-hints}
\end{figure}


%While it might seem that this hint ma solve the entire problem, 
Note that this hint is not immediately usable. Some of the challenges
for this example are:

{\bf Challenge 1.}  Although it may seem as if
method \code{get} is static allowing us to make an immediate call,
it is actually an instance method, requiring us to have an object of
class \code{Calendar}. However, no such object is available in the
original code meaning that it must be created by the refactored code.
Consequently, we must populate the component library with the necessary
components to create such an object by consuming existing objects.

{\bf Challenge 2.} Adding too many components to the library will
make the synthesis task unfeasible. In particular, we should
be able to differentiate between components that we can use
(we have or are able to generate
all the necessary objects for calling them), and those
that we can't because we can't obtain some of the arguments
and/or receiver. Adding the latter to the library will
significantly slow down the synthesis process by
adding infeasible programs to the search space.

{\bf Challenge 3.} Besides adding components that allow
us to generate the required objects, we might also need
components for initialising the new objects. For instance,
in Figure~\ref{ex:deprecated-method-other}, we must call
\code{calendar.setTime(date)} to set the calendar's date
based on the existing \code{date} object.

In Section~\ref{sec:components-seeding}, we provide in-depth details about the
seeding of the component library and tackling the above challenges. 
At a high level, the process is guided by types and code hints.

Once the component library is built, for the actual synthesis,
we use a counterexample-guided iterative process that enables us
to keep refining a candidate refactoring until we are happy with it.
The refinement is based on the results provided by 
coverage-guided property-based testing~\cite{DBLP:conf/issta/PadhyeLS19}.
%% Intuitively, we wanted to avoid the overheads due to program analysis and
%% constraint solving incurred by whitebox methods, while also making use of
%% knowledge about the code that can be obtained through lightweight instrumentation.
%% For this purpose, %% we phrased the synthesis problem in a manner that makes it amenable
%% %% to solving via coverage-guided testing. Essentially,
%% we employ an iterative synthesis process based on CounterExample Guided Synthesis (CEGIS)~\ref{},
%% where the problems posed in each iteration are phrased such that they can be solved
%% by coverage-guided property-based testing.
We provide more details about %
the synthesis process in Section~\ref{sec:encoding}.

%% Whitebox approaches can be inefficient as they require fully interpreting the
%% internal structures of the code through some form of static analysis.

%% In this section, we use the example in
%% Figure~\ref{ex:deprecated-method-other} to informally illustrate our
%% technique.  Firstly, our synthesis method is component based, meaning
%% that it uses components from a given library to generate new code.  We
%% use a type-directed method for generating a specifically designed
%% component library for each refactoring. We will provide more details
%% on how to generate the component library in
%% Section~\ref{sec:components-seeding}.  In the current section, we will
%% just assume the component library containts all the instructions
%% necessary to generate the refactoring in
%% Figure~\ref{ex:deprecated-method-other}.

%% Similar to CEGIS, our synthesis process is iterative such that in each
%% iteration an inductive synthesiser generates a candidate refactoring
%% that is subsequently checked by a verifier.  
%% In general, each CEGIS
%% verification phase produces {\em one} counterexample that is added to
%% the set of inputs used by the subsequent synthesis iteration to
%% generate a better candidate.  The implicit assumption is that these
%% inputs are somehow important such that generating a candidate that
%% works for them will help efficiently prune the search space and avoid
%% enumerating all possible programs.

%% In this paper, we propose using the notion of coverage in order to
%% generate counterexamples to be added to the inputs set. Basically, in
%% each iteration of the synthesis process, the verifier will generate
%% multiple counterexample inputs that cause the current candidate to fail the property of interest by
%% exercising different paths.
%% %
%% As we generate refactorings, the property that we are interested to check is observational equivalence
%% with respect to the original code, i.e. for each input, the refactoring must generate the
%% same result as the original code (in reality, as we must handle code with side effects,
%% the refactored and original codes must generate the same program state). 

%% For our running example, the synthesiser generates the following candidate
%% refactoring in the first synthesis iteration:
 
%% \begin{lstlisting}[mathescape=true,showstringspaces=false]
%% void main(String[] args) {
%%   int hour = java.util.Calendar.getInstance().
%%              get(java.util.Calendar.HOUR_OF_DAY);
%% }
%% \end{lstlisting}

%% Compared to the original program in Figure~\ref{ex:deprecated-method-other}, this
%% candidate refactoring is not semantically equivalent as it does not make
%% use of the initial \texttt{date} variable. When passed to the verifier,
%% we obtain X counterexamples, each exercising a different path through the
%% candidate refactoring (note that, while these paths are not visible in the
%% code above, they come from the methods being called). With all these counterexamples,
%% the correct refactoring is found in the second iteration.

%% Our conjecture is that coverage-guided counterexamples help prune
%% the search space during CEGIS. In the experimental evaluation, we will
%% empirically evaluate our conjecture.

%%==========

%% Our technique relies on both syntax and semantics and makes use of advances in the
%% field of program synthesis. In particular, we make use of the general
%% structure of Counterexample Guided Inductive Synthesis (CEGIS), but
%% instead of using symbolic execution to both generate and verify
%% candidate refactorings, we use fuzzing, as described later in this
%% section.  Our synthesis method is also component based, meaning that
%% it uses components from a given library to generate new code.  In the
%% rest of this section, we describe how we build the components library
%% followed by presenting the actual refactoring synthesis.
%%Our technique for refactoring deprecated instances (e.g. fields, methods, classes) has two big steps, as described next.

%% \subsection{{\bf Input to our method}}
%% We expect as input the original deprecated code \var{Orig} with inputs
%% $\vec{x}$, as well as access to the library where the
%% field/method/class has been deprecated. We need the latter to seed the
%% components library.  Depending on the user's intention we may also
%% take into consideration some amount of context (of the location where
%% the deprecated element is found) when searching for a refactoring.

\section{{\bf Code hints and type guided seeding of the component library}}\label{sec:components-seeding}
% Our synthesis engine is component based, meaning that, in order to generate new code, it makes use of a library of components.
As mentioned in Section~\ref{sec:overview}, a critical aspect of our technique is building the
component library. 
%For each required refactoring, we start by building the components library.
%This step is critical as
Notably, %% the size of the component library has a direct impact
%% on the size of the search space for the respective refactoring.
if the component library is too small, meaning that it doesn't contain all
the required components, then the synthesis will fail to find a refactoring.
Similarly, if we add too many components, then the search space becomes too large and
synthesis may not terminate within a reasonable amount of time.

Consequently, in our approach, the component library is dynamically built for each refactoring,
such that it only contains components specific to that use case.
This simplifies the job if the subsequent synthesis task.
In order to increase our chances of finding a refactoring fast, we start with a restricted code  {\em core library} and incrementally increase its size
until we find a refactoring.


\paragraph{The core component library}
As mentioned before, we make use of type and code hints to build the component library.
%Initially, we build our components library for the actual synthesis process.
%
%% In order to restrict the number of components that get added to the library,
%% it is straightforward to make use of type information.
%% Types have been used extensively in program synthesis, including for
%% component-based approaches~\cite{}.
%
In the example in Figure~\ref{ex:deprecated-method-other}, we can view \code{date} of type \code{Date} as the input to the code to be refactored. The objective is to weave together components that consume this object and generate the integer \code{hour}. While there are many methods in the Java standard library that return an integer without consuming an object of type $Date$, our intention is to only add those that allow using the input object. %% Basically, we have  
%% \code{source\_obj}=\{\code{date}\} with given type \code{source\_type}=\{\code{Date}\}. We need to consume this in order to obtain an object of type \code{target\_type}=\{\code{int}\}.

%% In general, when deprecating a field/method/class, the
%% \code{@deprecated} Javadoc tag is used in the comment section to
%% inform the developer the reason of deprecation and what can be used in
%% place. This enables us to pick components directly from the comments accompanying the
%% \code{@deprecated} tag.

% Other github projects that have already been ported.
%\todo{Maybe we can also collect sketches as especially the unit tests will provide us with more than just the name of the replacement.}
% As opposed to the comments in libraries, the unit tests are syntactically valid

%% As an example, the \code{getHours} method in class \code{Date} used in Figure~\ref{ex:deprecated-method-other} is accompanied by the following comment:

%% \begin{lstlisting}[mathescape=true,showstringspaces=false]
%% /**
%%  * Returns the hours represented by this <code>Date</code>
%%  * object as an integer between 0 and 23.
%%  *
%%  * @return the hours represented by this date object.
%%  * @deprecated Use Calendar instead of Date,
%%  * and use get(Calendar.HOUR_OF_DAY)
%%  * instead.
%%  * @see Calendar
%%  * @see #setHours(int)
%%  */  
%% \end{lstlisting}

%% /**
 %% * Creates a new Date Object representing the given time.
 %% *
 %% * @deprecated use <code>new GregorianCalendar(year+1900, month,
 %% * day)</code> instead.
 %% * @param year the difference between the required year and 1900.
 %% * @param month the month as a value between 0 and 11.
 %% * @param day the day as a value between 0 and 31.
 %% */

With respect to the code hints, we refer to those in Figure~\ref{ex:code-hints}.
Intuitively, the library seeding algorithm must add components that are able to consume the
inputs to the deprecated code and produce the objects required to make use
of the Javadoc hints. %% For our example, the only object that can be seen as an input for the 
%% deprecated code is \code{Source\_Obj}=\{\code{date}\} with given type \code{Source\_Type}=\{\code{Date}\}. Using this, we need to obtain an object of type \code{Target\_Type}=\{\code{Calendar}\}.
In this case, the code hints instruct us to add method \code{``int get(int field)''} from class
\code{Calendar} and constant \code{``Calendar.HOUR_OF_DAY''} to our library.
The difficulty is that, in order to be able to call method \code{get}, we require an object of
type \code{Calendar} and our library doesn't contain any way of creating such an object.
Therefore, we must seed our component library with ways of generating such an object.

We first scan all the public constructors of class \code{Calendar}
and all the public methods from class \code{Calendar} that return an object of type \code{Calendar}. We find the following four options:

\begin{enumerate}
  \item \code{static Calendar	getInstance()} -- gets a calendar using the default time zone and locale.
  \item \code{static Calendar getInstance(Locale aLocale)} -- gets a calendar using the default time zone and specified locale.
  \item \code{static Calendar	getInstance(TimeZone zone)} -- gets a calendar using the specified time zone and default locale.
  \item \code{static Calendar	getInstance(TimeZone zone, Locale aLocale)} -- gets a calendar with the specified time zone and locale.
\end{enumerate}

%% Seeding the library is crucial in obtaining an efficient synthesis technique. Thus,
%% we want to only add instructions that can be used. For this purpose, we start by collecting the given types and the objects to consume from the deprecated code. %% Using these, we need to create an object of the required type.
%% For our illustrating example, the only object used by the deprecated code is \code{Source\_Obj}=\{\code{date}\} with given type \code{Source\_Type}=\{\code{Date}\}. Using these, we need to obtain an object of type \code{Target\_Type}=\{\code{Calendar}\}, so that we can make use of the JavaDoc hint.  
%
Out of the four options above, we search for those for which we have the means to generate all their arguments and their receiver. This means that we are only interested in constructors/methods
where the types of the arguments and receiver are included in \code{source\_types}.
%% Basically, the components library needs to contain constructors/methods for creating objects whose types
%% match the types of the arguments and the receiver, or constants of those types.
%
We call such an instruction {\em realizable}, according to Definition~\ref{def:realizable}.
In the definition, we use
%\var{ensured\_types(i)} to denote the set of the types corresponding to the objects that can be
%generated using instruction \var{i}, and
\var{required\_types(i)} to denote
the set of the types of the parameters and receiver required to call instruction \var{i}.

\begin{definition}[Realizable instruction]\label{def:realizable}
Instruction \var{i} is {\em realizable} iff $\forall t \in required\_types(i). t \in source\_types$.
  
\end{definition}

Regarding the four methods in our example, the first one is realizable. Conversely, in order to call the second method, we would need to generate an object of type \code{Locale},
for which purpose we don't have an instruction in the library.
The last two methods are in a similar situation as the second. For the core library, we only add the first method.

%% Even though we add them, they are not going to be used by the synthesis engine
%% as we use a type-directed approach to term generation...

The seeding algorithm for the core library is provided in Figure~\ref{alg:seeding-core}.
We start by adding all the constants contained in the Javadoc suggestion
and in the original code to the library. Then, we also add all the instructions contained
in the Javadoc suggestion. Next, we need to make sure that the Javadoc suggestions
are realizable. For this purpose, we add ways of generating
objects whose types match the types of the required fields and receivers.

Basically, for each required type that we have no way of generating, we add all the realizable
public constructors and realizable public methods returning objects of that type to the
core library.
Notably, it may be the case that there is no realizable instruction for a required type.
In such a case, the synthesiser won't be able to generate any useful programs.
More details on this in Section~\ref{sec:synthesis}.

Note that, once we added the needed components to generate an object of a given type,
we add that type to $Source\_Types$.
%% we overload \var{ensured\_types} to take \var{library} as an argument, in which case it returns the set of all the types of the objects that can be created using the instructions in the library.

%% In certain situations, the seeding of the component library may require additional work.
%% For instance, for the example in Figure~\ref{ex:deprecated-method-other}, we need to also provide a way of obtaining
%% a \code{Calendar} object. Thus, we add \code{Calendar.getInstance()}. 

\SetKwInOut{KwIn}{Input}
\SetKwInOut{KwOut}{Output}

\begin{figure}
%% \begin{algorithm}[H]
%% \SetAlgoLined
%% \KwOut{Core component library}
%%  Add constants suggested under the @deprecated tag to the $library$\;
%%  Add constants used in the original code to the $library$\;
%%  Add instructions suggested under the @deprecated tag to the library\;
%%  \For{instr $\in$ library}{
%%    \If{$\neg$realizable(instr)}{
%%      \For{$type {\in} required\_types(instr) ~\&\&~ type {\not\in} Source\_Types$}{
%%        $generators = realizable\_public\_constructors(type) \cup realizable\_public\_methods\_returning\_object\_of\_type(type)$\;
%%        Add $generators$ to the library\;
%%        \If{$generators \neq \emptyset$}{$Source\_types = Source\_types \cup\{type\} $}
%%      }
%%    }
%%  }
%% \end{algorithm}
 \caption{Seeding algorithm for the core library}
\label{alg:seeding-core}
\end{figure}

\paragraph{Incremental expansion of the library}
If we fail to find a refactoring within a certain amount of time,
we start to incrementally increase the size of the library by adding additional components to be used during synthesis.
The library expansion algorithm is provided in Figure~\ref{alg:seeding-extended}.
Essentially, we go through all the instructions in the current library and add
ways of generating objects that can be used as their arguments (i.e. whose types match that of their arguments).
Note that, as opposed to the core library, here we look at all instructions, not just the realizable ones.

%% Similarly to the core library, we start by adding all the constants contained in the Javadoc suggestion
%% and in the original code, as well as all the instructions contained
%% in the Javadoc suggestion to the library. Next, we need to make sure that the Javadoc suggestions
%% are realizable. We add all the instructions generating
%% objects whose types are required for this purpose to the library.
%% This is contrary to the core library, where we were only adding realizable instructions.
%% Moreover, in a second \code{for} loop, we attempt to make all the instructions currently in
%% the library realizable by adding required (this time only realizable) instructions.

%% The seeding algorithm for the core library is provided in Figure~\ref{alg:seeding-extended}.

\begin{figure}
%% \begin{algorithm}[H]
%% \SetAlgoLined
%% \KwIn{Current library}
%% \KwOut{Expanded component library}
%%  \For{instr $\in$ library}{
%%      \For{$type {\in} required\_types(instr)$}{
%%        $generators = realizable\_public\_constructors(type) \cup realizable\_public\_methods\_returning\_object\_of\_type(type)$\;
%%        Add $generators$ to the library\;
%%      }
%%  }
%% \end{algorithm}
 \caption{Incremental expansion of the component library}
\label{alg:seeding-extended}
\end{figure}


%% \begin{figure}
%% \begin{algorithm}[H]
%% \SetAlgoLined
%% \KwResult{Core component library}
%%  Add constants suggested under the @deprecated tag\;
%%  Add constants used in the original code\;
%%  %Add constants 0 and 1 to the library\;
%%  %Add + to the library\;
%%  Add instructions suggested under the @deprecated tag to the library\;
%% ; 
%%  \For{t $\in$ required-types(library)}{
%%    \If{t $\not\in$ ensured-types(library)}{
%%      generators = public-constructors(t) $\cup$ public-methods-returning-an-object-of-type(t)\;
%%      Add ``generators'' to the library\;
%%      \For{t $\in$ required-types(generators)}{
%%        \If{t $\not\in$ ensured-types(library)}{
%%          new-generators = realizable-public-constructors(t) $\cup$ realizable-public-methods-returning-an-object-of-type(t)\;
%%          Add ``new-generators'' to the library\;
%%        }
%%      }
%%    }
%%  }
%% \end{algorithm}
%%  \caption{Seeding algorithm for the extended library}
%% \label{alg:seeding-extended}
%% \end{figure}



As future work we aim to also collect components and sketches from the  unit tests for the updated library.
 
\paragraph{{\bf Automatically generating the desired refactoring}}
\todo{Counterexample guided type-directed synthesis?}

For this purpose, we use a counterexample-guided iterative process that enables us to keep refining a candidate refactoring until we are happy with it. This process consists of two phases, described next:
  \begin{itemize}
  \item[{\bf Phase 1:}] Given a fixed set of inputs $In$, find a refactoring that works for $In$. For this purpose, we encode   
    the synthesis problem as a verification problem that can be solved by fuzzing. We discuss this problem encoding in
    Section~\ref{sec:encoding}. The found refactoring is called a candidate.
  \item[{\bf Phase 2:}] Given the candidate synthesised at the previous step, verify whether it is indeed a correct refactoring.
    Again, we answer this question by encoding it as a software verification problem to be solved by fuzzing (see Section~\ref{sec:encoding}).
    If an input for which the candidate isn't observationally equivalent to the original code is found, then
    we return to the previous step and add this input to the set $In$. Otherwise we have found a correct refactoring and we are done.
  \end{itemize}

\todo{Run through the motivational example: candidates, counterexamples etc.}  



\section{Graybox synthesis}\label{sec:encoding}

In this section, we discuss our encoding of the problem solved in each
of the two phases of the synthesis process.  We refer to the original
code as \var{Orig(\vec{x})} and the final refactored code as
\var{Refactor(\vec{x})}, where we assume that both \var{Orig} and
\var{Refactor} have the same inputs \var{\vec{x}}. Moreover,
we use \var{Candidate(\vec{x})} to refer to a candidate refactoring.

%% We encode the refactoring problem as a verification problem that can be solved by
%% fuzzing. We next describe the verification and synthesis phases in a CEGIS module,
%% where \var{Orig} and \var{Refactor} denote the original and refactored code,
%% respectively.

\paragraph{The verification phase} For the verification phase, we are provided with a candidate
refactoring and we must check whether there exists any input
\var{\vec{x}} for which the original code and the candidate
refactoring are not observationally equivalent.  To do this, we build
the following \code{Verify} method, which given some input
\var{\vec{x}}, asserts that the two programs are equivalent for
\var{\vec{x}}.

\begin{lstlisting}[mathescape=true,showstringspaces=false]
Verify(x) {
  assert(equivalent(Orig(x), Candidate(x)));
}
\end{lstlisting}

In other words, answering the initial question posed by the verification
phase is reduced to checking the safety of this method: if
\code{Verify} is safe (i.e. the assertion is not violated) for any
input \var{\vec{x}}, then there is no input that can distinguish
between the original and the candidate refactoring meaning that this
is indeed a sound refactoring. However, if \code{Verify} is not safe,
then we want to be able to obtain a counterexample input
\var{\vec{x_{cex}}} for which the assertion fails. This counterexample
will be provided to the synthesis phase and used to refine the current
candidate. We check the safety of \code{Verify} with coverage-guided
fuzz testing. While this means that we may fail to find a
distinguishing input when one does exist, we discuss later in the
paper our reasons for choosing to use fuzzing over other sound
verification techniques.

A core part of the verification phase and the whole synthesis process
is the \var{equivalent} procedure, which checks that the original code
and a candidate refactoring are equivalent for a given input.
Naively, this would be checked by comparing the results returned by the
\var{Orig} and \var{Candidate}. However, that is not enough as
we must also consider the heap representation. We will discuss this in
Section~\ref{sec:equiv}.

%% notion of program equivalence.  While this is undecidable in
%% general, our synthesis procedure in the verification phase this problem is restricted to
%% checking that \var{Orig} and \var{Candidate} are equivalent for a .


%% The fuzzer will find an input for which the original and refactored code
%% produce a different result.

\paragraph{The synthesis phase} For the synthesis phase, we are provided with a
a finite set of input examples $\{x_1 \cdots x_n\}$ and we need to find a candidate refactoring
that is observationally equivalent to the original program for these inputs.
For this purpose, we construct the following \code{Synthesise} method, which
takes the refactored code \var{Candidate} as input.

\begin{lstlisting}[mathescape=true,showstringspaces=false]
Synthesise (Candidate) {

  if (equivalent(Orig($x_1$), Candidate($x_1$)) && ...
  && equivalent(Orig($x_n$), Candidate($x_n$)))
   assert(false);
}
\end{lstlisting}

Again, we reduce the problem to be solved in the synthesis phase to
the problem of checking the safety of the \code{Synthesise} method,
which we do by fuzzing.
The fuzzer will find an input \var{Candidate} that fails the assertion,
meaning that it returns the same output as \texttt{Orig}
on the finite set of inputs \var{x_1, \cdots, x_n}.


%% - Performing the synthesis. Complex bit: synthesising method invocations for methods that contain loops.

%% \begin{enumerate}
%% \item We seed the templates/snippets for the synthesis library using the following two strategies:
%%   \begin{itemize}
%% \item Use deprecated block messages.
%% \item Use the unit tests for the new library. As opposed to the comments in libraries,
%%   the unit tests are syntactically valid.
%%   \end{itemize}

%% \item Use the seeded templates to generate the refactoring.
%%   \begin{itemize}
%% \item Look at checking equivalence in the presence of loops.
%%   \end{itemize}
%%   \end{enumerate}

\subsection{Checking program equivalence for a certain input}\label{sec:equiv}

%\paragraph{Checking heap equivalence}

In Section~\ref{sec:encoding}, we use \var{equivalent(Orig(x),
  Candidate(x))} to denote equivalence checking for \var{Orig} and
\var{Candidate} on concrete input \var{x}.
In this section, we provide details on how we check this equivalence.


%% First, this function needs
%% to check whether for input \var{x}, the outputs of the original and
%% candidate refactored programs are the same.  Moreover, any realistic
%% Java program makes modifications to the heap. Consequently,
%% \var{equivalent} must check that heap is being handled in the same
%% manner in both programs.


We'll generally refer to the original code as \var{P_1} and the refactored one
as \var{P_2}. We further use:
\begin{itemize}
\item \var{loadedClasses(P)} to obtain the set of classes loaded by the class loader
  in which the execution of \var{P} is performed.
\item \var{liveVars(P)} returns the set of variables that are live
at the end of the $P$. %% Thus, \var{liveVars(P_1)} gives us the
%% variables that have been updated in the original code \var{P_1} and are
%% live at the end of \var{P_1}.
\item \var{staticFields(C)} returns
  the set of static fields of a class \var{C}.
\end{itemize}  

\begin{example}\label{ex:defs}
  For our running example in Figure~\ref{ex:deprecated-method},
  \var{P_1} is represented by the following line of code:

\begin{lstlisting}[mathescape=true,showstringspaces=false]
  Date date = new Date(120, 12, 30);
\end{lstlisting}

Whereas \var{P_2} is denoted by:

\begin{lstlisting}[mathescape=true,showstringspaces=false]
  final Calendar calendar = Calendar.getInstance();
  calendar.set(2020, 12, 30);
  final Date date = calendar.getTime();
\end{lstlisting}
  
Then, we have:
\[
\begin{aligned}[t]
  \var{liveVars(P_1)} &= \{date\}\\
  \var{liveVars(P_2)} &= \{calendar, date\}\\  
  \var{loadedClasses(P_1)} &= \{Date\} \\
  \var{loadedClasses(P_2)} &= \{Calendar, Date\} \\  
  \var{staticFields(Date)} &= \emptyset\\
  \var{staticFields(Calendar)} &= \{DATE, YEAR, \cdots\}  
\end{aligned}
\]
\end{example}


In order to extract the value assigned to a variable \var{v} by the execution of \var{P} on a specific input \var{x},
we will use the notation \var{E[P(x)](v)}. Essentially, if we consider the trace generated by executing \var{P(x)},
then \var{E[P(x)]} maps each variable defined in \var{P} to the latest value assigned to it by this trace.


Next, we define the notion of equivalence with respect to a concrete input \var{x}.

\begin{definition}[Program equivalence with respect to a concrete input \var{x} {\bf[partial]}]\label{def:prog-equiv}
  Given two programs \var{P_1} and \var{P_2} and concrete input \var{x},
  we say that \var{P_1} and \var{P_2} are equivalent
  with respect to \var{x}, written as \var{equivalent(P_1(x), P_2(x))}
  if and only if the following conditions hold:

\[
    \begin{aligned}
      & (1)~ \forall v\in liveVar(P_1)\cap liveVar(P_2). equivalent(E[P_1(x)](v), E[P_2(x)](v))\\
%& (2) equivalent(aliasEquivClass(liveVar(P_1)), aliasEquivClass(liveVar(P_2)))      
& (2)~  \forall C \in loadedClasses(P_1) \cap loadedClasses(P_2).\forall f \in staticFields(C).\\
  &  ~~~~~ equivalent(E[P_1(x)](f), E[P_2(x)](f))
    %%   throws an exception and   $equiv(exception(R_1), exception(R_1))$. 
  %% \item \var{R_1} doesn't throw an exception and $equiv(R_1,R_2)$ and
   %% $\forall f_i \in getStatic(class(R_1)) \wedge g_i \in getStatic(class(R_2)). equiv(f_i,g_i)$.
    \end{aligned}
    \]
   
  \end{definition}

The above definition says that in order for \var{P_1} and
\var{P_2} to be equivalent with respect to input \var{x}, any variable
that is live at the end of both \var{P_1} and \var{P_2} must be
assigned equivalent values by the executions of \var{P_1} and
\var{P_2} on \var{x}, respectively.
Additionally, for any class that
is being loaded by the class loader of both \var{P_1} and \var{P_2},
all the static fields must again be assigned equivalent values by the
two executions. %% Note that we use the notation \var{v/P(x)} to denote the
%% value assigned to \var{x} by the execution of \var{P} on input \var{x}.
Note that, as mentioned in its name, Definition~\ref{def:prog-equiv} only partially
defines the equivalence of \var{P_1} and \var{P_2} on input \var{x}. For now
we will ignore this and get back to it later in the section.

\begin{example}\label{ex:equiv}
  When applying Definition~\ref{def:prog-equiv} to \var{P_1} and \var{P_2} given in Example~\ref{ex:defs},
  the following conditions must hold:
\[
    \begin{aligned}
& (1)~ \forall v\in \{date\}. equivalent(E[P_1(x)](v), E[P_2(x)](v))\\
& (2)~  \forall C \in \{Date\}.\forall f \in staticFields(C).equivalent(E[P_1(x)](f), E[P_2(x)](f))
    %%   throws an exception and   $equiv(exception(R_1), exception(R_1))$. 
  %% \item \var{R_1} doesn't throw an exception and $equiv(R_1,R_2)$ and
   %% $\forall f_i \in getStatic(class(R_1)) \wedge g_i \in getStatic(class(R_2)). equiv(f_i,g_i)$.
    \end{aligned}
    \]

    Given that class \var{Date} has no static fields, the second condition is trivially met. Then, in order for the two portions of code to be
    equivalent with respect to an input \var{x}, the objects assigned to variable \var{date} by the execution of  \var{P_1(x)} and \var{P_2(x)}
    must be equivalent. We will discuss checking equivalence of two objects in two distinct executions next.
    
\end{example}

\todo{define $equivalent(E[P_1(x)](v), E[P_2(x)](v))$}.

\paragraph{Preserving aliasing}
When expressing the equivalence relation between \var{P_1} and \var{P_2}, we intentionally missed one important aspect,
namely aliasing. To understand the problem let's look a the following example, which denotes a slightly
modified version of the running example.

\begin{example}\label{ex:aliasing}
\begin{lstlisting}[mathescape=true,showstringspaces=false]
  Date date1 = new Date(120, 12, 30);
  Date date2 = new Date(120, 12, 30);  
  
  final Calendar calendar = Calendar.getInstance();
  calendar.set(2020, 12, 30);
  Date date1 = calendar.getTime();
  Date date2 = date1;
\end{lstlisting}

For the code above, we will refer to the first two lines denoting the
original deprecated code as \var{P_3}, and the rest, denoting the
refactored code, as \var{P_4}.  We note that, as opposed to the
running example, \var{P_3} defines two variables \var{date1} and
\var{date2}, both assigned a \var{Date} object created by calling the
\var{Date} constructor with identical arguments.
Conversely, in \var{P_4}, \var{date1} and \var{date2} are aliases, i.e. they point to the same
memory location.

Note that, according to Definition~\ref{def:prog-equiv}, given an input \var{x}, \var{P_3} and \var{P_4} are equivalent with respect to \var{x}.
\todo{I should use a concrete input here.}
So, everything looks good until now. Let's however, assume that the following code
is used after \var{P_3/P_4}, corresponding to whether we refer to the original or refactored code.

\begin{lstlisting}[mathescape=true,showstringspaces=false]
  date1 = new Date(120, 10, 30);
\end{lstlisting}

Our intention is to execute the whole code, including this additional line, on input \var{x}.
If \var{P_3} and \var{P_4} were indeed semantically equivalent with respect to input \var{x},
then we would expect \var{date1} and \var{date2} to have the same value at the end of both
executions, respectively. However, this is not the case.
In the original code, using \var{P_3}, \var{date2} will be assigned an object
denoting the date ``30/12/2020'', whereas in the refactored code,
\var{date2} will correspond to the date ``30/10/2020''. This is due to the fact that,
in \var{P_4}, \var{date1} and \var{date2} are aliases. Then, when the object referenced by
\var{date1} was changed to refer to the date denoting ``30/10/2020'', this also affected \var{date2}.
Thus, \var{P_3} and \var{P_4} are not semantically equivalent for input \var{x}.
\end{example}  

Intuitively, any aliases between live variables at the end of the original code should also
be present at the end of the refactored code. For this purpose, we use the notation
\var{aliasEquivClass(V)} which returns the set of all equivalence classes induced over the
set of variables \var{V} by the aliasing relation.  
Notably, the same aliasing equivalence relation must also hold over the
static fields of the classes loaded by both programs.


\begin{example}
For \var{P_1} and \var{P_2}, there are no aliases, therefore:
\[
\begin{aligned}[t]
  \var{aliasEquivClass(liveVar(P_1))} &= \emptyset\\
  \var{aliasEquivClass(liveVar(P_2))} &= \emptyset
\end{aligned}
\]
%
However, for \var{P_3} and \var{P_4} we have:
\[
\begin{aligned}[t]
  \var{aliasEquivClass(liveVar(P_3))} &= \emptyset\\
  \var{aliasEquivClass(liveVar(P_4))} &= \{\{date1, date2\}\}
\end{aligned}
\]
In \var{P_4}, the alising relation induces one equivalence class, namely \var{\{date1, date2\}}.

\end{example}
  
Next, we adjust Definition~\ref{def:prog-equiv} to capture the aliasing aspect.

\begin{definition}[Program equivalence with respect to a concrete input \var{x} {\bf[complete]}]\label{def:prog-equiv}
  Given two programs \var{P_1} and \var{P_2} and concrete input \var{x},
  we say that \var{P_1} and \var{P_2} are equivalent
  with respect to \var{x}, written as \var{equivalent(P_1(x), P_2(x))}
  if and only if the following conditions hold:

\[
    \begin{aligned}
      & (1)~ \forall v\in liveVar(P_1)\cap liveVar(P_2). equivalent(E[P_1(x)](v), E[P_2(x)](v))\\
& (2)~ \forall v_1,v_2 \in liveVar(P_1) \cap liveVar(P_2). aliases(P_1, v_1, v_2) \Rightarrow aliases(P_2, v_1,v_2)\\      
& (3)~  \forall C \in loadedClasses(P_1) \cap loadedClasses(P_2).\forall f \in staticFields(C).\\
      &  ~~~~~ equivalent(E[P_1(x)](f), E[P_2(x)](f))\\
      & (4)~\forall f_1,f_2 \in staticFields(loadedClasses(P_1) \cap loadedClasses(P_2)).\\
      & ~~~~~ aliases(P_1, f_1, f_2) \Rightarrow aliases(P_2, f_1,f_2)
    %%   throws an exception and   $equiv(exception(R_1), exception(R_1))$. 
  %% \item \var{R_1} doesn't throw an exception and $equiv(R_1,R_2)$ and
   %% $\forall f_i \in getStatic(class(R_1)) \wedge g_i \in getStatic(class(R_2)). equiv(f_i,g_i)$.
    \end{aligned}
    \]
   
  \end{definition}

where, given a program \var{P}, variables \var{v_1} and \var{v_2} are aliases, i.e. \var{aliases(P, v_1, v_2)} holds,
if and only if they are in the same equivalence class induced by the aliasing relation,
  i.e. \var{aliasEquivClass(\{v_1,v_2\}) = \{\{v_1,v_2\}\}}.

  For the fourth condition, we abuse the notation to use \var{staticFields} over a set of classes, rather than just one class.
  The objective is to return the set of all static fields defined in all the classes in the set of classes taken as argument.
  
\todo{revise the rest.}
\todo{also add an example where static fields are changed.}

Given the potential use of aliases, we will use a Hash table to
associate a symbolic id to each pointer variable such that two
variable share the same id if and only if they are aliases.
%% Then, we can check whether the set of aliases is preserved in the
%% refactored program. For instance, if \texttt{obj1} and \texttt{obj2}
%% are aliases in the original program, then they must be aliases also
%% in the second program.
\todo{check exactly what this is used for.}

Note on aliasing checking:
Everything that is part of our comparison is in the isolated class loader.
For instance user defined types will usually be located in the isolated
class loader, enabling the comparison.
However, what do we do about wrapper object aliasing?
For anything outside the class loader, we use reference equality.
\todo{example?}

We execute the two programs on the given input and then compare the
result for equivalence. The two programs are equivalent with respect to
the given input if one of the conditions below are fulfilled: 
\begin{enumerate}
\item They throw the same exception.
\item They return equivalent values and all the static fields in the two corresponding classes
  are respectively equivalent.
\end{enumerate}  

Let's refer to the two programs as \var{P_1} and \var{P_2}, and the corresponding execution results
as \var{R_1} and \var{R_2}. We use $class(R)$ to get the class of object \var{R} and
$getStatic(C)$ to retrieve the static fields of \var{C}. Then, we provide the definition
for program equivalence on input \var{i} as follows:

\begin{definition}[Program equivalence with respect to a concrete input]
  Given the results \var{R_1} and \var{R_1} of executing \var{P_1} and \var{P_2}
  on the input \var{i}, respectively, we say the \var{P_1} and \var{P_2} are equivalent
  with respect to \var{i} if and only if one of the following holds:

  \begin{itemize}
  \item \var{R_1} throws an exception and   $equiv(exception(R_1), exception(R_1))$. 
  \item \var{R_1} doesn't throw an exception and $equiv(R_1,R_2)$ and
    $\forall f_i \in getStatic(class(R_1)) \wedge g_i \in getStatic(class(R_2)). equiv(f_i,g_i)$.
    \end{itemize}
    
  \end{definition}

  
Next, we define equivalence between two objects, denoted above by \var{equiv(R_1,R_2)}.

Equivalence classes induced by the aliasing relation need to be preserved by the
refactored program.



%% Aliasing check: checks whether the two given objects have the same aliasing ID, and thus
%%    alias with the same objects on their respective heaps.

%%    What do you assign id's to? I'm trying to figure out whether this is too strict?
%%    Basically, equivalence classes induced by the aliasing relation need to be preserved,
%%    but they may not have the same representative ID.

%% You compare execution results:
%% - do you just look at the returned object?

%% Equivalent execution results
%% - returned value
%% - static fields of the corresponding classes

%% Equivalent objects
%% - ObjectIdComparator
%% - IsAliasingEquivalent

%% State: static fields of all classes, stack vars, follow object pointers, thread locals?, equivalence classes induced by the aliasing relation must be preserved
%% - touch the same vars
%% - all touched vars correspond to equivalent objects
%% - 2 graphs are isomorphic

%% 2 obj are equal if all their fields are equal
%% 2 primitive types are equal if they are the same value

%% ============

%% a1 = new Integer(1);
%% b1 = new Integer(1);

%% vs

%% a2 = new Integer(1);
%% b2 = a2;




\subsubsection{Loaded classes}

One of the potential issues when checking equivalence of two program
is the treatment of loaded classes. In particular, one program might
load more classes than the other, but still have the exact same effect
as the other one. This can happen for instance if one program calls a
helper method from a class that needs to be loaded, whereas the other
program inlines the same method, thus avoiding loading the class.  Our
assumption is that loadeding and initialization don't perform any
changes outside of the observable state already checked by the
existing equivalence check.  While this assumption is generally safe
to make, there might be some cases when this is not the case, e.g. the
class initializer might print something on the screen. For now, we
don't handle these cases.

%% A class is initialized when a symbol in the class is first used. When
%% a class is loaded it is not initialized.

\todo{give example}



Discuss different types of fuzzing and the one that we use.
\todo{Double check with Pascal what he uses.}

%% \section{Algorithm}

%% \todo{Write description of the algorithm}

%% \subsection{Reduction of semantically equivalent programs}

\section{Guarantees}

\paragraph{Discussion on soundness} Given that we rely on fuzzing, we can't guarantee that the
refactoring works for all possible inputs. However, it is well suited
to the particular scenarios for which we designed our technique,
IDE integration where the refactoring is presented as a suggestion to the programmer,
or automated pull requests to be approved by the developers.
%% where
%% continuous integration practices are in place. In particular, once our
%% technique has found a possible refactoring, we sumit a pull request
%% that needs to be approved by other developers.

In order to fully guarantee the soundness of the refactorings proposed
by our technique, we would have to use a sound verification technique
in the verification phase mentioned earlier. However, such a technique
would require symbolic execution, which is notoriously difficult for
real-world programs due to features such as loops, library functions,
complex data structures. The most successful such verification
techniques are either bounded (i.e. check that an erroneous state
cannot be reached within a bounded number of loop unrollings) or
require user input in the form of loop invariants. While the former
class of techniques provides limited soundness guarantees, the latter
requires intricate help from the user.  In fact, we are not aware of
any sound and automatic symbolic verification method that can handle
the kind of code we are interested in.

Discuss tradeoff: symbolic execution is expensive; programming
languages evolve faster than verification techniques and therefore
many new features are not supported by their front ends (and/or
backends depending on how complex the new feature is); even after
translating the code into a logical formula, calls to solvers are
expensive.

\paragraph{Discussion on completeness}



%% \section{Some research questions}

%% What if the new method doesn't maintain observational equivalence?
%% TODO: Check existing projects to see how many do preserve equivalence.
%% For the rest, is there a notion of partial equivalence that ensures the equivalence
%% for the new projects minus the differences introduced by the new function.

%% How easy is for the fuzzer to handle such structured inputs as we need in the
%% synthesis phase?


%% \subsection{For the experimental evaluation}
%% \begin{itemize}
%% \item Pick libraries that have deprecated instances.
%% \item Pick users of these libraries (i.e. open source projects that are actively maintained).
%% \item Make pull requests and see how many get merged.
%% \end{itemize}  

%% \section{Context-sensitive vs. context-insensitive refactoring}

%% \paragraph{Context-insensitive refactoring}
%% We consider different refactoring options. The first and most simple
%% one is the so called ``context-insensitive'' refactoring, where we
%% find a refactoring for the deprecated code that works irrespective of
%% context. This means that, once such a refactoring is found, it can be
%% used to replace all the uses of the deprecated code, e.g. a deprecated
%% method invocation will be replaced with the same refactored code
%% irrespective of the actual parameters or rest of the state at the call
%% site.

%% For instance, if we want to refactor the call to method
%% \texttt{toRefactor} in the code below, then we search for a refactoring that is
%% observationally equivalent to the given method irrespective of the
%% context of the call.

%% \begin{lstlisting}[mathescape=true]
%% class NoContext {
%%   int x;

%%   public double toRefactor(int y, int z) {
%%     return (double) x + y + z;
%%   }
%% }
%% \end{lstlisting}
 
%% This means that the state that we can vary in order to check for
%% observational equivalence includes \texttt{x, y} and \texttt{z}.

%% \paragraph{Context-sensitive refactoring}
%% The refactoring context consists of the context where the code to be refactored
%% is placed. For instance, in the snippet below, if we are trying to refactor the call to the deprecated
%% function \texttt{f}, then the context gives us the valuation $[\texttt{x}\mapsto 1]$ for the function's argument. 

%% \begin{lstlisting}[mathescape=true]
%%   int x = 1;
%%   A a = new A();
  
%%   a.f(1)
%% \end{lstlisting}
    
%% We now look at the possibility of considering the refactoring context
%% when searching for a refactoring. This gives us the option of starting
%% with very little context and gradually increasing it until finding a
%% refactoring. \todo{Adjust this based on what the experiments do.}

%% The other option is to find a refactoring that takes into consideration
%% the calling context. For instance, in the code below we have two
%% calls to \texttt{toRefactor}, one in method \texttt{foo} and one in
%% method \texttt{bar}. When considering the calling context, we have to
%% find two different refactorings, one for each of the method calls.
%% For the call in \texttt{foo}, the state that we need to consider for observational
%% equivalence contains \texttt{ctx} and \texttt{a}, whereas for the call in method
%% \texttt{bar} it contains \texttt{ctx, a, b, c, d} and \texttt{e}.

%% \begin{lstlisting}[mathescape=true]
%%   class Context1 {
%%     NoContext ctx = new NoContext();

%%     public double foo(int a) {
%%       return ctx.toRefactor(a, a);
%%     }
%%   }

 

%%   class Context2 {
%%     NoContext ctx = new NoContext();

%%     int d;
%%     int e;

%%     public double bar(int a, int b, int c) {
%%       return ctx.toRefactor(a, b + c + d + e);
%%     }
%%   }
%% \end{lstlisting}


%% Fuzzing:

%% someNoContext.x = x_*;

%% someNoContext.toRefactor(y_*, z_*);

%% Possible follow-up:
%% - if we have a sizeable codebase, the user might have to do 1,2 deprecated
%% refactorings and then the synthesisers can use the PRs done by the user
%% to seed the synthesiser and do the other thousands of refactorings
%% automatically.

\section{Implementation}

\subsection{Program fuzzing}


In order to generate programs, we make use of the fuzzing technique described
in~\cite{DBLP:conf/issta/PadhyeLS19} called JQF.  This technique is
coverage-guided, which means that it uses lightweight program
instrumentation to trace the code coverage reached by each generated
input that is fed to the program under test.  Basically, the program
is continuously executed with randomly generated inputs. However,
instead of generating inputs from scratch, coverage-guided fuzzers
evolve a set of saved inputs. If a new input leads to an increase
in code coverage, it is saved for subsequent evolution.

In our setting, the randomly generated inputs are programs.
This is a difficult setting for fuzzing because programs are
highly structured, whereas fuzzers often work with unstructured inputs
that can be generated via byte-level mutations such as bit flips.

JQF is designed to handle structured inputs, where inputs of type \code{T}
are generated by a backing \code{Generator<T>}.
While JQF provides a library of such generators, we manually write our own.
Essentially, our generator describes the grammar corresponding to the
programs to be synthesised.
JQF never generates the same exact program twice. However it may generate many
syntactically different, but semantically equivalent ones.
Next, we discuss a few techniques that we use to reduce the number of
generated programs that are semantically equivalent.

\todo{more details about the generators}

\subsection{Instrumentation}

The goal of our instrumentation is to execute a program candidate in
the same context as the original, deprecated method. The original
method can affect the program state by virtue of its return value,
field assignments in the case of instance methods, and static field
assignments. In order to account for these effects, our
instrumentation passes a reference to the original method's object
instance, if present, as an argument to the program candidate.
An example of this is presented in
Fig.~\ref{ex:side-effects-instrumentation}.

\begin{figure}
\begin{lstlisting}[mathescape=true,showstringspaces=false]
// Redundant snippet:
Date today;
Date tomorrow;
  
void init(int year, int month, int day) {
  // Original:
  this.setDates(year, month, day);

  // Instrumented:
  Program.of(0xabcd).execute(year, month, day, this);
}
  
@Deprecated
private void setDates(int year, int month, int day) {
  final Calendar calendar = Calendar.getInstance();
  calendar.set(year, month, day);
  this.today = calendar.getTime();
  calendar.add(Calendar.DAY_OF_MONTH, 1);
  this.tomorrow = calendar.getTime();
}
\end{lstlisting}
\caption{Deprecated side-effects instrumentation.}
\label{ex:side-effects-instrumentation}
\end{figure}

Fig.~\ref{ex:date-instrumentation} illustrates an example where no
such relevant instance object exists and thus only the original
arguments are passed as arguments into the program candidate.

\begin{figure}
\begin{lstlisting}[mathescape=true,showstringspaces=false]
// Redundant snippet:
static void main(String[] args) {
  // Original:
  Date date = new Date(120, 12, 30);

  // Instrumented:
  Date date = Program.of(0xabcd).execute(120, 12, 30);
}
\end{lstlisting}
\caption{Deprecated `Date' instrumentation.}
\label{ex:date-instrumentation}
\end{figure}

\subsection{Abstract classes and interfaces}

\todo{talk about Mockito}

\subsection{Parsing the @code hints}

\todo{use the context of the method to be refactored}

\section{Experimental results}\label{sec:experimental-results}

\section{Semantically different refactorings}


\begin{figure}
\begin{lstlisting}[mathescape=true,showstringspaces=false]
void main(String[] args) {
  // Deprecated:
  Date date = new Date(120, 12, 30);
 
  // Should have been:
  final Calendar calendar = Calendar.getInstance();
  calendar.set(2020, 12, 30);
  final Date date = calendar.getTime();

  // Or:
  final Date date = new GregorianCalendar(2020, 12, 30)
                      .getTime();
}
\end{lstlisting}
\caption{Deprecated method example.}
\label{ex:deprecated-method}
\end{figure}


Sometimes, the deprecated and the new features are not semantically
equivalent on the entire domain. Reasons for this might be that a bug
was fixed, a certain functionality was improved, or simply the fact
that the deprecated feature is only expected to be used on restricted
parts of the input domain. In such a case, the deprecated and the new
feature may not be semantically equivalent on the entire input domain.

For illustration, let's consider the example in Figure~\ref{ex:three-dates}.
In the example, we make use of the constructor \code{new Date(year, month, day)}, 
which is deprecated with the 
recommendation to use the \code{Calendar} class instead.
The refactoring is not trivial given that the 
\code{Date} constructor expects its year parameter to be an offset
from 1900.  This needs to be
transformed by adding 1900 when using \code{Calendar} API, as shown in the
example. Moreover, as already seen in the motivational example in  Figure~\ref{ex:deprecated-method-other},
the \code{Calendar} constructor is protected,
meaning that we must use \code{getInstance} to obtain a
\code{Calendar} object.
Alternatively, the \code{GregorianCalendar} class may be used with
similar difficulties.
In the example, we use \var{date1}, \var{date2} and \var{date3} to denote the three distinct ways of constructing a date.

%% For illustration, let's consider the three distinct ways of
%% constructing a date captured in Figure~\ref{ex:three-dates} and
%% denoted by \var{date1}, \var{date2} and \var{date3}.

\begin{figure}
  \begin{lstlisting}[mathescape=true,showstringspaces=false]
  // Deprecated:
  Date date1 = new Date(year, month, day);

  // Should have been:
  Calendar calendar = Calendar.getInstance();
  calendar.set(1900 + year, month, day, 0, 0, 0);
  calendar.set(Calendar.MILLISECOND, 0);
  Date date2 = calendar.getTime();
    
  // Or:
  Date date3=new GregorianCalendar(1900+year,month,day).getTime();
  \end{lstlisting}
\caption{Deprecated `Date' example.}
\label{ex:three-dates}
\end{figure}

%% In the example, \var{date1} uses the deprecated
%% \var{Date(year, month, day)} constructor, whereas \var{date2} and
%% \var{date3} make use of the recommended \var{Calendar} and
%% \var{GregorianCalendar} classes.
Notably, the ways in which \var{date2} and \var{date3} are computed
represent a valid refactoring for the piece of code computing
\var{date1}. However, when we provide this example to our
verifier (as described in Section~\ref{X}), we obtain the following
counterexample:

  \begin{lstlisting}[mathescape=true,showstringspaces=false]
    int year = -89412298;
    int month = 1439435067;
    int day = 378993182;
  \end{lstlisting}

  For these values of \var{year}, \var{month} and \var{day}, the three
  dates evaluate to:

  \begin{lstlisting}[mathescape=true,showstringspaces=false]
    date1: "Sat May 02 00:00:00 CEST 31580172"
    date2: "Fri Jul 02 00:00:00 CEST 31580799"
    date3: "Fri Jul 02 00:00:00 CEST 31580799"
  \end{lstlisting}

  Where \var{date1} evaluates to a different date than \var{date2} and \var{date3}. In principle, this should not affect any user as the given
  \var{year}, \var{month} and \var{day} are clearly artificial and outside the expected domain of a calendaristic date.
  In particular, for this example, we deduced that the negative year is the cause for the different values of \var{date1}, \var{date2} and \var{date3}.

  This counterexample is enough to stop our synthesiser from finding any of the correct refactorings in Figure~\ref{ex:three-dates}.
  To overcome this problem, we plan on computing preconditions for the verification process. Namely, for this example, we would collect all the concrete values that
  \var{year}, \var{month} and \var{day} get instantiated to in the unit tests associated with the given project, and use them to
  infer a precondition over the domain of these three variables. For \var{year}, this should tell us that the expected year must be positive.
  Consequently, the verifier won't attempt to find counterexamples involving negative years, thus eliminating the counterexample above.


  
\section{Related Work}

Related paper: in \cite{DBLP:conf/paste/Perkins05}, the authors
replace calls to deprecated methods by their bodies.

Often, genetic improvement
\cite{DBLP:journals/dagstuhl-reports/PetkeGFL18} is used for the
purpose of code refactoring. Due to the manner in which such works measure
the fitness of potential refactorings, genetic improvement cannot vary
the refactoring context.  Conversely, our technique allows varying this
context until a refactoring is being found.

\paragraph{Program refactoring}

Cheung et al.~describe a system that automatically transforms fragments of
application logic into SQL queries~\cite{DBLP:conf/pldi/CheungSM13}. 
Moreover, similar to our approach, the authors rely on synthesis technology
to generate invariants and postconditions that validate their
transformations (a~similar approach is presented
in~\cite{DBLP:conf/cc/IuCZ10}).  The main difference (besides the actual
goal of the work, which is different from ours) to our work is that the
lists they operate on are immutable and do not support operations such as
remove.  Capturing the potential side effects caused by such operations is
one of our work's main challenges.

Syntax-driven refactoring base program transformation decisions
on observations on the program's syntax tree.  Visser
presents a purely syntax-driven framework~\cite{stratego}.  The
presented method is intended to be configurable for specific
refactoring tasks, but cannot provide guarantees about semantics
preservation.  The same holds for~\cite{txl} by Cordy et al.,
\cite{sawin} by Sawin et al., \cite{bae} by Bae et al.~and
\cite{chris} by Christopoulou et al.  In contrast to these approaches,
our procedure constructs an equivalence proof before transforming the
program. In \cite{conf/sigsoft/GyoriFDL13}, Gyori et al. present a
similar refactoring to ours but performed in a syntax-driven manner.
%
Steimann et al.~present Constraint-Based Refactoring in \cite{Steimann2011},
\cite{Steimann2012Pilgrim} and \cite{Steimann2011KollePilgrim}. Their approach
generates explicit constraints over the program's abstract syntax tree to
prevent compilation errors or behaviour changes by automated refactorings.
% This gives rise to a flexible framework of customisable refactorings,
% implementable through a refactoring constraint specification language
% (cf. \cite{Steimann2011KollePilgrim}).
The approach is limited by the information
a program's AST provides and thus favours conservative implementations of
syntax-focused refactorings such as \emph{Pull Up Field}.
%
Fuhrer et al.~implement a type constraint system to introduce missing type
parameters in uses of generic classes (cf. \cite{DBLP:conf/ecoop/FuhrerTKDK05})
and to introduce generic type parameters into classes which do not provide
a generic interfaces despite being used in multiple type contexts
(cf. \cite{DBLP:conf/icse/KiezunETF07}).
%
% Raychev et al.~present a semi-automatic approach where users perform
% incomplete refactorings manually and then employ a constraint solver
% to find a sequence of default refactorings such as move or rename
% which include the users' changes. The engine is limited to syntactic
% matching with the users' partial changes and does not consider program
% semantics~\cite{DBLP:conf/oopsla/RaychevSSV13}.
%
% Weissgerber and Diehl rely on meta information to classify changes
% between software versions as refactorings~\cite{weiss}.  The technique
% aims to identify past refactorings performed by programmers, but is
% not a decision procedure for automated refactorings.
%
O'Keffe and Cinn{\'{e}}ide present search-based
refactoring~\cite{search1, search2}, which is similar to syntax-driven
refactoring.  They rephrase refactoring as an optimisation problem,
using code metrics as fitness measure.  As such, the method optimises
syntactical constraints and does not take program semantics into
account.
%
% Bavota et al. implement refactoring decisions in \cite{Bavota:2011:IEC}
% using semantic information limited to identifiers and comments,
% which may differ from the actual semantics (e.g. due to bugs).
%
Kataoka et al. interpret program semantics to apply refactorings
\cite{Kataoka:2001:ASP:846228.848644}, but use dynamic test execution
rather than formal verification, and hence their transformation lacks
soundness guarantees.
%
Franklin et al. implement a pattern-based refactoring approach
transforming statements to stream queries~\cite{Gyori:2013:CGI:2491411.2491461}.
Their tool LambdaFicator~\cite{DBLP:conf/icse/FranklinGLD04} is available as a
NetBeans branch. We compared \tool against it in our experimental evaluation
in Sec.~\ref{experiments-results}.
%
%% \paragraph{Heap Logics}
%
%% While many decidable heap logics have been developed recently, none are
%% expressive enough to capture operations allowed by the Java Collection
%% interface, operations allowed by the Java Stream interface as well as
%% equality between collections (for lists this implies that we must be able to
%% reason about both content of lists and the order of
%% elements)~\cite{DBLP:conf/cav/ItzhakyBINS13, DBLP:conf/cav/PiskacWZ13,
%% DBLP:conf/esop/BrainDKS14, DBLP:conf/popl/MadhusudanPQ11,
%% DBLP:conf/atva/BouajjaniDES12, DBLP:conf/lpar/DavidKL15}.  On the other
%% hand, very expressive transitive closure
%% logics~\cite{DBLP:conf/csl/ImmermanRRSY04} are not concise and easily
%% translatable to stream code.

\paragraph{Program synthesis}

An approach to program synthesis very similar to ours is Syntax Guided
Synthesis (SyGuS)~\cite{sygus}.  SyGuS synthesisers supplement the logical
specification with a syntactic template that constrains the space of allowed
implementations.  Thus, each semantic specification is accompanied by a
syntactic specification in the form of a grammar.  Other second-order
solvers are introduced in~\cite{DBLP:conf/pldi/GrebenshchikovLPR12,
DBLP:conf/cav/BeyenePR13}.  As opposed to ours, these focus on
Horn clauses.

\section{Conclusion}

We conjecture that refactorings driven by the semantics of programs have
broader applicability and are able to address more complex refactoring
schemata in comparison to conventional syntax-driven refactorings, thereby
increasing the benefits of automated refactoring.  The space of possible
semantic refactoring methods is enormous; as an instance, we have presented
a method for refactoring iteration over Java collection classes based on
program synthesis methods.  Our experiments indicate that refactoring using
this specific instance is feasible, sound and sufficiently performant. 
Future research must broaden the evidence for our general hypothesis by
considering other programming languages, further, ideally more complex
refactoring schemata, and other semantics-based analysis techniques.

%\acks

% We recommend abbrvnat bibliography style.
\bibliographystyle{ACM-Reference-Format}
%\bibliographystyle{abbrv}
\bibliography{document}



\end{document}

%                       Revision History
%                       -------- -------
%  Date         Person  Ver.    Change
%  ----         ------  ----    ------

%  2013.06.29   TU      0.1--4  comments on permission/copyright notices

